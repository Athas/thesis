\chapter{First Order Optimisations}
\label{chap:first-order-optimisations}

As a data-parallel programming language, most of the interesting
optimisations for \LO{} naturally revolve around SOACs.  Yet,
classical optimisations such as copy propagation, constant folding,
hoisting and common subexpression elimination remain important.  This
chapter will cover their implementation for \LO{}.

It turns out that hoisting and common subexpression elimination can be
unified in a single framework, which in the \LO{} compiler is termed
the \textit{Rebinder}.  It is described in \cref{sec:rebinder}.

\section{Inlining}

One property currently holds for all optimisations performed by the
\LO{} compiler: they are all strictly intraprocedural.  Thus, we rely
on aggressive inlining as the first step of optimisation, wherein we
inline every non-recursive function call.  Inlining a large function
at multiple call sites can of course result in a tremendous amount of
code bloat, but as function calls are in any case usually always
inlined on current GPU hardware, due to no stack being available, this
is perhaps excusable.

After inlining, most functions will be dead, and are summarily
removed.

\section{Let- and tuple-normalisation}
\label{sec:let-normalisation}

At its core, program optimisation is about recognising code patterns,
and rewriting them to a more efficient form that retains the meaning
of the original code.  To make this process simpler, we pre-process
the program to give it a more regular structure.  The use of internal
\LO{} as presented in \cref{chap:internal} is an important step in
this process, but it is not sufficient by itself.  To this end, we use
a transformation pass that rewrites needlessly complex program
structure into a simpler form.  The mechanics behind the
transformation are tedious and unimportant (basically a recursive
traversal through the syntax tree), and it is best understood by the
invariants guaranteed of the resulting program:

\begin{itemize}
\item Tuple expressions can appear only as the final result of a
  function, SOAC, or \texttt{if} expression, and similarly for the
  tuple pattern of a let binding, e.g., a formal argument cannot be a
  tuple,
\item Consecutive \texttt{let}, \texttt{let-with} and \texttt{loop}
  expressions are at the same nesting level, e.g., $e_1$ cannot be a
  \texttt{let} expression when used in
  \texttt{let~$p$~=~$e_1$~in~$e_2$},
\item Each \texttt{if} is bound to a corresponding \texttt{let}
  expression, and an \texttt{if}'s condition cannot be in itself an
  \texttt{if} expression, e.g.,
\begin{center}
\begin{colorcode}
a + if (if c\cindx{1} then \(e\myindx{1}\) else \(e\myindx{2}\))
    then \(e\myindx{3}\)
    else \(e\myindx{4}\)
  \(\Downarrow\)
let c\(\myindx{2}\) = if c\(\myindx{1}\)then \(e\myindx{1}\) else \(e\myindx{2}\) in
let b = if c\(\myindx{2}\) then \(e\myindx{3}\) else \(e\myindx{4}\) in a+b
\end{colorcode}
\end{center}
\item Function calls, including SOACs, have their own \texttt{let}
  binding, e.g., \texttt{g(reduceT(f,a))} $\Rightarrow$
  \texttt{let~y~=~reduceT(f,e,a)~in~g(y)},
\item All actual arguments in a function call are vars, e.g.,
  \texttt{f(a+b)}$\Rightarrow$\texttt{let~x=a+b~in~f(x)}.
\end{itemize}

Note that we consider ``function-like'' constructs such as
\texttt{transpose}, \texttt{reshape} and \texttt{replicate} to be
functions as far as the above invariants are considerned.

\section{Copy/constant propagation and constant folding}

Copy propagation is the mechanism by which we eliminate bindings that
are merely copies of existing variables.  Constant propagation is the
inlining of constant bindings where the bindings are
used. Constant-folding is the process of evaluating a constant
expression at compile time, for example an addition where both
operands are statically known.
\Cref{fig:copy/constant-propagation/folding} illustrates the
difference between the three processes.

\begin{figure}
\begin{subfigure}[t]{.33\textwidth}
\centering
\begin{colorcode}
let x = 2 in
let y = 3 in
let z = x in
z + y
  \(\Downarrow\)
let x = 2
let y = 3 in
x + y
\end{colorcode}
\subcaption{Copy propagation}
\end{subfigure}%
\begin{subfigure}[t]{.33\textwidth}
\centering
\begin{colorcode}
let x = 2 in
let y = 3 in
x + y
  \(\Downarrow\)
2 + 3
\end{colorcode}
\vspace{2.9\baselineskip}
\subcaption{Constant propagation}
\end{subfigure}%
\begin{subfigure}[t]{.33\textwidth}
\centering
\begin{colorcode}
2 + 3
  \(\Downarrow\)
5
\end{colorcode}
\vspace{4.7\baselineskip}
\subcaption{Constant folding}
\end{subfigure}
\caption{Examples of copy/constant propagation and constant folding}
\label{fig:copy/constant-propagation/folding}
\end{figure}

In imperative compilers, these optimisations are usually performed on
a program after it has been converted to a basic block graph.
However, after undergoing the let/tuple-normalisation described in the
previous section, it is easy to perform all three optimisations in
tandem directly on the syntax tree of an \LO{} program.

The central idea is that we consider some expressions to be
\textit{inlineable}.  Whenever the expression in the bindee position
of a \texttt{let}-expression is inlineable, we substitute any
occurrences of the name bound by the binding within the body of the
\texttt{let}-expression by the bindee expression (we ignore bindings
where the pattern is a tuple-pattern for now).  For example, consider
this expression:
\begin{colorcode}
let a = 2 in
\(e\)
\end{colorcode}
We consider the expression \texttt{2} (a constant number) to be
inlinable, hence we substitute \texttt{2} for \texttt{a} within $e$
and remove the binding of \texttt{a} entirely.

A big question is which expressions to inline.  As inlining may
duplicate the inlined expression, we should only inline where such
duplication will not result in additional computation at run-time.  As
a start, we can certainly inline variables and non-array constants.
Incidentally, this by itself provides copy- and constant-propagation.

As for other expressions, we note that \texttt{reshape} and
\texttt{transpose} operations are entirely index-space
transformations, and can thus be handled at compile-time wherever the
result of the operation is used.  Although uninplemented in the
current \LO{} compiler, it is envisioned that a transformation similar
to the one on \cref{fig:removing-reshape-transpose} will be used by
the code generator.  Therefore, we freely inline \texttt{reshape} and
\texttt{transpose}.  As \texttt{iota} and \texttt{replicate} can be
removed in a similar manner, they are therefore also considered
inlineable.

Bindings with tuple-patterns can be inlined under some circumstances,
specifically if the bindee is itself a tuple literal, where every
component is an inlineable expression.

\begin{figure}
\centering
\begin{subfigure}[t]{.4\textwidth}
\centering
\begin{colorcode}
let b = transpose(a) in
b[i,j]
  \(\Downarrow\)
a[j,i]
\end{colorcode}
\end{subfigure}%
\begin{subfigure}[t]{.5\textwidth}
\centering
\begin{colorcode}
let b = reshape((n,m), a) in
b[i,j]
  \(\Downarrow\)
a[i*m+j] // Assuming 'a' has rank 1.
\end{colorcode}
\end{subfigure}%
\caption{Removing \texttt{reshape} and \texttt{transpose}}
\label{fig:removing-reshape-transpose}
\end{figure}

However, even if an expression is in principle inlineable, there are
still three cases that prevent inlining:

\begin{itemize}
\item If an array-typed variable is indexed, we need to keep it in the
  program, unless the replacement expression is itself a variable
  (that is, copy propagation).  This is because \LO{} only permits
  indexing of variables, not arbitrary array-typed expressions.

\item If an array-typed variable is used as the source in a
  \texttt{let-with}, we again need to keep its binding in the program.

\item If a variable cannot be substituted with an expression for some
  other reason (notably, because it would violate the
  \texttt{let}-normalisation properties from
  \cref{sec:let-normalisation}), we also cannot remove its binding.
\end{itemize}

Apart from substituting variables, we also look at other expressions
to determine whether constant folding is possible. This is done by a
bottom-up traversal of the syntax tree, where each expression is
processed as follows:

\begin{description}
\item[\texttt{\textit{x} \textbf{binop} \textit{y}}] \hfill\\
  If \textit{x} and \textit{y} are literals, compute and substitute
  with the result.

  \item[\texttt{\textbf{unop} \textit{x}}]\hfill\\
    If \textit{x} and \textit{y} is a literal, compute and substitute
  with the result.

  \item[\texttt{size($k$, \textit{a})}]\hfill\\
    Depending on $k$ and how \textit{a} was bound, we may be able to
    replace the \texttt{size} expression.
    \begin{itemize}
    \item If $k=0$ and \texttt{a} is the result of \texttt{iota(n)} or
      \texttt{replicate(n,e)}, we can substitute with simply
      \texttt{n}, as that is the size of \texttt{a}.
    \item If \texttt{a} is a literal constant, we can substitute
      with the exact size.
    \end{itemize}

  \item[\texttt{let \textit{pat} = \textit{e} in \textit{body}}]\hfill\\
    If, after transforming \textit{body}, none of the names in
    \textit{pat} are used, remove the binding.

  \item[\texttt{if \textit{c} then \textit{a} else \textit{b}}]\hfill\\
    If \textit{c} can be constant-folded to either \texttt{True} or
    \texttt{False}, replace with the corresponding branch.

  \item[\texttt{\textit{f}(...)}]\hfill\\
    If all parameters to a function call are literal values, we use
    the interpreter to evaluate the function and insert its return
    value.  At the moment, we assume that the function will terminate,
    although this assumption is not really justifiable.  Instead, we
    should probably only evaluate non-recursive functions.

  \item[\texttt{a[i]}]\hfill\\
    There are several potential avenues for constant-folding index
    operations, depending on how \texttt{a} was bound:
    \begin{description}
    \item[\texttt{a} is bound to a variable \texttt{b}:] Replace with \texttt{b[i]}.

    \item[\texttt{a} is \texttt{iota(n)}:] Replace with \texttt{i} --
      note that this may make an invalid program valid, as we remove
      the bounds checks $\texttt{i}<\texttt{n}$.  We could use the
      assertion mechanism from \cref{sec:assertions} to bring it back,
      but this is not done in the current compiler.

    \item[\texttt{a} is \texttt{b[j]}:] Replace with \texttt{b[j,i]}.

    \item[\texttt{a} is an array literal:] Replace with the
      corresponding element in the array literal.

    \item[\texttt{a} is \texttt{replicate(n, v)}:] Replace with
      \texttt{v}.
    \end{description}

  \item[\texttt{a[i,j]}]\hfill\\
    If more than one index is given, we can handle the same cases as
    above (although indexing into e.g. an \texttt{iota} would of
    course be a type error), as well as a few more:\footnote{For
      simplicity, I treat only the case where two indices are given.
      The implementation in the \LO{} compiler supports an arbitrary
      number indices.}
    \begin{description}
    \item[\texttt{a} is \texttt{transpose(b)}:] Replace with
      \texttt{b[j,i]}.
    \item[\texttt{a} is \texttt{replicate(n,iota(m)):}] Replace with
      \texttt{j}.
    \end{description}
\end{description}

\section{Hoisting}

In the compiler literature, \textit{hoisting} (also called
\textit{loop-invariant code motion} by those textbook authors paid per
page) is the movement of loop-invariant expressions out of a loop.
This has a clear benefit: rather than executing once per iteration of
the loop, the expression is executed once before the loop begins.
When compiling an imperative language, we must be careful not to move
any code with side effects, but in a pure language such as \LO{}, we
can hoist freely (with a few restrictions that I'll get to in
\cref{sec:when-not-to-hoist}).  A simple example of hoisting in action
is shown on \cref{fig:simple-hoisting}.

\begin{figure}
\begin{center}
\begin{bcolorcode}
map(fn int (int x) =>
      let k = y + z in
      x + k,
    a)
\end{bcolorcode}
\hspace{1cm}
\adjustbox{valign=t}{
\(\Rightarrow\)
}%
\hspace{1cm}
\begin{bcolorcode}
let k = y + z in
map(fn int (int x) =>
      x + k,
    a)
\end{bcolorcode}
\end{center}

\caption{Hoisting in action}
\label{fig:simple-hoisting}
\end{figure}

\subsection{When not to hoist}
\label{sec:when-not-to-hoist}

\section{Common Subexpression Elimination}

\section{The Rebinder}
\label{sec:rebinder}

%%% Local Variables: 
%%% mode: latex
%%% TeX-master: "thesis.tex"
%%% End: 
