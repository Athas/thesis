\chapter{First Order Optimisations}
\label{chap:first-order-optimisations}

As a data-parallel programming language, most of the interesting
optimisations for \LO{} naturally revolve around SOACs.  Yet,
classical optimisations such as copy propagation, constant folding,
hoisting and common subexpression elimination remain important.  This
chapter will cover their implementation for \LO{}.

It turns out that hoisting and common subexpression elimination can be
unified in a single framework, which in the \LO{} compiler is termed
the \textit{Rebinder}.  It is described in \cref{sec:rebinder}.

\section{Inlining}

One property currently holds for all optimisations performed by the
\LO{} compiler: they are all strictly intraprocedural.  Thus, we rely
on aggressive inlining as the first step of optimisation, wherein we
inline every non-recursive function call.  Inlining a large function
at multiple call sites can of course result in a tremendous amount of
code bloat, but as function calls are in any case usually always
inlined on current GPU hardware, due to no stack being available, this
is perhaps excusable.

After inlining, most functions will be dead, and are summarily
removed.

\section{Let- and tuple-normalisation}
\label{sec:let-normalisation}

At its core, program optimisation is about recognising code patterns,
and rewriting them to a more efficient form that retains the meaning
of the original code.  To make this process simpler, we pre-process
the program to give it a more regular structure.  The use of internal
\LO{} as presented in \cref{chap:internal} is an important step in
this process, but it is not sufficient by itself.  To this end, we use
a transformation pass that rewrites needlessly complex program
structure into a simpler form.  The mechanics behind the
transformation are tedious and unimportant (basically a recursive
traversal through the syntax tree), and it is best understood by the
invariants guaranteed of the resulting program:

\begin{itemize}
\item Tuple expressions can appear only as the final result of a
  function, SOAC, or \texttt{if} expression, and similarly for the
  tuple pattern of a let binding, e.g., a formal argument cannot be a
  tuple,
\item Consecutive \texttt{let}, \texttt{let-with} and \texttt{loop}
  expressions are at the same nesting level, e.g., $e_1$ cannot be a
  \texttt{let} expression when used in
  \texttt{let~$p$~=~$e_1$~in~$e_2$},
\item Each \texttt{if} is bound to a corresponding \texttt{let}
  expression, and an \texttt{if}'s condition cannot be in itself an
  \texttt{if} expression, e.g.,
\begin{center}
\begin{colorcode}
a + if (if c\cindx{1} then \(e\myindx{1}\) else \(e\myindx{2}\))
    then \(e\myindx{3}\)
    else \(e\myindx{4}\)
  \(\Downarrow\)
let c\(\myindx{2}\) = if c\(\myindx{1}\)then \(e\myindx{1}\) else \(e\myindx{2}\) in
let b = if c\(\myindx{2}\) then \(e\myindx{3}\) else \(e\myindx{4}\) in a+b
\end{colorcode}
\end{center}
\item Function calls, including SOACs, have their own \texttt{let}
  binding, e.g., \texttt{g(reduceT(f,a))} $\Rightarrow$
  \texttt{let~y~=~reduceT(f,e,a)~in~g(y)},
\item All actual arguments in a function call are vars, e.g.,
  \texttt{f(a+b)}$\Rightarrow$\texttt{let~x=a+b~in~f(x)}.
\end{itemize}

Note that we consider ``function-like'' constructs such as
\texttt{transpose}, \texttt{reshape} and \texttt{replicate} to be
functions as far as the above invariants are considerned.

\section{Copy/constant propagation and constant folding}

Copy propagation is the mechanism by which we eliminate bindings that
are merely copies of existing variables.  Constant propagation is the
inlining of constant bindings where the bindings are
used. Constant-folding is the process of evaluating a constant
expression at compile time, for example an addition where both
operands are statically known.
\Cref{fig:copy/constant-propagation/folding} illustrates the
difference between the three processes.

\begin{figure}
\begin{subfigure}[t]{.33\textwidth}
\centering
\begin{colorcode}
let x = 2 in
let y = 3 in
let z = x in
z + y
  \(\Downarrow\)
let x = 2
let y = 3 in
x + y
\end{colorcode}
\subcaption{Copy propagation}
\end{subfigure}%
\begin{subfigure}[t]{.33\textwidth}
\centering
\begin{colorcode}
let x = 2 in
let y = 3 in
x + y
  \(\Downarrow\)
2 + 3
\end{colorcode}
\vspace{2.9\baselineskip}
\subcaption{Constant propagation}
\end{subfigure}%
\begin{subfigure}[t]{.33\textwidth}
\centering
\begin{colorcode}
2 + 3
  \(\Downarrow\)
5
\end{colorcode}
\vspace{4.7\baselineskip}
\subcaption{Constant folding}
\end{subfigure}
\caption{Examples of copy/constant propagation and constant folding}
\label{fig:copy/constant-propagation/folding}
\end{figure}

In imperative compilers, these optimisations are usually performed on
a program after it has been converted to a basic block graph.
However, after undergoing the let/tuple-normalisation described in the
previous section, it is easy to perform all three optimisations in
tandem directly on the syntax tree of an \LO{} program.

The central idea is that we consider some expressions to be
\textit{inlineable}.  Whenever the RHG of a \texttt{let}-expression is
inlineable, we substitute any occurrences of the name bound by the
binding within the body of the \texttt{let}-expression by the RHS (we
ignore bindings where the pattern is a tuple-pattern for now).  For
example, consider this expression:
\begin{colorcode}
let a = 2 in
\(e\)
\end{colorcode}
We consider the expression \texttt{2} (a constant number) to be
inlinable, hence we substitute \texttt{2} for \texttt{a} within $e$
and remove the binding of \texttt{a} entirely.

A big question is which expressions to inline.  As inlining may
duplicate the inlined expression, we should only inline where such
duplication will not result in additional computation at run-time.  As
a start, we can certainly inline variables and non-array constants.
Incidentally, this by itself provides copy- and constant-propagation.

As for other expressions, we note that \texttt{reshape} and
\texttt{transpose} operations are entirely index-space
transformations, and can thus be handled at compile-time wherever the
result of the operation is used.  Although uninplemented in the
current \LO{} compiler, it is envisioned that a transformation similar
to the one on \cref{fig:removing-reshape-transpose} will be used by
the code generator.  Therefore, we freely inline \texttt{reshape} and
\texttt{transpose}.  As \texttt{iota} and \texttt{replicate} can be
removed in a similar manner, they are therefore also considered
inlineable.

Bindings with tuple-patterns can be inlined under some circumstances,
specifically if the RHS is itself a tuple literal, where every
component is an inlineable expression.

\begin{figure}
\centering
\begin{subfigure}[t]{.4\textwidth}
\centering
\begin{colorcode}
let b = transpose(a) in
b[i,j]
  \(\Downarrow\)
a[j,i]
\end{colorcode}
\end{subfigure}%
\begin{subfigure}[t]{.5\textwidth}
\centering
\begin{colorcode}
let b = reshape((n,m), a) in
b[i,j]
  \(\Downarrow\)
a[i*m+j] // Assuming 'a' has rank 1.
\end{colorcode}
\end{subfigure}%
\caption{Removing \texttt{reshape} and \texttt{transpose}}
\label{fig:removing-reshape-transpose}
\end{figure}

However, even if an expression is in principle inlineable, there are
still three cases that prevent inlining:

\begin{itemize}
\item If an array-typed variable is indexed, we need to keep it in the
  program, unless the replacement expression is itself a variable
  (that is, copy propagation).  This is because \LO{} only permits
  indexing of variables, not arbitrary array-typed expressions.

\item If an array-typed variable is used as the source in a
  \texttt{let-with}, we again need to keep its binding in the program.

\item If a variable cannot be substituted with an expression for some
  other reason (notably, because it would violate the
  \texttt{let}-normalisation properties from
  \cref{sec:let-normalisation}), we also cannot remove its binding.
\end{itemize}

Apart from substituting variables, we also look at other expressions
to determine whether constant folding is possible. This is done by a
bottom-up traversal of the syntax tree, where each expression is
processed as follows:

\begin{description}
\item[\texttt{\textit{x} \textbf{binop} \textit{y}}] \hfill\\
  If \textit{x} and \textit{y} are literals, compute and substitute
  with the result.

  \item[\texttt{\textbf{unop} \textit{x}}]\hfill\\
    If \textit{x} and \textit{y} is a literal, compute and substitute
  with the result.

  \item[\texttt{size($k$, \textit{a})}]\hfill\\
    Depending on $k$ and how \textit{a} was bound, we may be able to
    replace the \texttt{size} expression.
    \begin{itemize}
    \item If $k=0$ and \texttt{a} is the result of \texttt{iota(n)} or
      \texttt{replicate(n,e)}, we can substitute with simply
      \texttt{n}, as that is the size of \texttt{a}.
    \item If \texttt{a} is a literal constant, we can substitute
      with the exact size.
    \end{itemize}

  \item[\texttt{let \textit{pat} = \textit{e} in \textit{body}}]\hfill\\
    If, after transforming \textit{body}, none of the names in
    \textit{pat} are used, remove the binding.

  \item[\texttt{if \textit{c} then \textit{a} else \textit{b}}]\hfill\\
    If \textit{c} can be constant-folded to either \texttt{True} or
    \texttt{False}, replace with the corresponding branch.

  \item[\texttt{\textit{f}(...)}]\hfill\\
    If all parameters to a function call are literal values, we use
    the interpreter to evaluate the function and insert its return
    value.  At the moment, we assume that the function will terminate,
    although this assumption is not really justifiable.  Instead, we
    should probably only evaluate non-recursive functions.

  \item[\texttt{assert(True)}]\hfill\\
    Replace with \texttt{Checked}.

  \item[\texttt{<$c_{1}$,\ldots,$c_{n}$>e}]\hfill\\
    Remove any certificate $c_{i}$ that is bound to \texttt{Checked}
    (i.e. a certificate that is always true).

  \item[\texttt{conjoin($c_{1}$,\ldots,$c_{n}$)}]hfill\\
    Remove any certificate $c_{i}$ that is bound to \texttt{Checked}
    (i.e. a certificate that is always true).

  \item[\texttt{a[i]}]\hfill\\
    There are several potential avenues for constant-folding index
    operations, depending on how \texttt{a} was bound:
    \begin{description}
    \item[\texttt{a} is bound to a variable \texttt{b}:] Replace with \texttt{b[i]}.

    \item[\texttt{a} is \texttt{iota(n)}:] Replace with \texttt{i} --
      note that this may make an invalid program valid, as we remove
      the bounds check $\texttt{i}<\texttt{n}$.  We could use the
      assertion mechanism from \cref{sec:assertions} to bring it back,
      but this is not done in the current compiler.

    \item[\texttt{a} is \texttt{b[j]}:] Replace with \texttt{b[j,i]}.

    \item[\texttt{a} is an array literal:] Replace with the
      corresponding element in the array literal.

    \item[\texttt{a} is \texttt{replicate(n, v)}:] Replace with
      \texttt{v}.
    \end{description}

  \item[\texttt{a[i,j]}]\hfill\\
    If more than one index is given, we can handle the same cases as
    above (although indexing into e.g. an \texttt{iota} would of
    course be a type error), as well as a few more:\footnote{For
      simplicity, we treat only the case where two indices are given.
      The implementation in the \LO{} compiler supports an arbitrary
      number of indices.}
    \begin{description}
    \item[\texttt{a} is \texttt{transpose(b)}:] Replace with
      \texttt{b[j,i]}.
    \item[\texttt{a} is \texttt{replicate(n,iota(m)):}] Replace with
      \texttt{j}.
    \end{description}
\end{description}

\section{Hoisting}
\label{sec:hoisting}

In the compiler literature, \textit{hoisting} (also called
\textit{loop-invariant code motion} by those textbook authors paid per
page) is the movement of loop-invariant expressions out of a loop.
This has a clear benefit: rather than executing once per iteration of
the loop, the expression is executed once before the loop begins.
When compiling an imperative language, we must be careful not to move
any code with side effects, but in a pure language such as \LO{}, we
can hoist freely (with a few restrictions that I'll get to in
\cref{sec:when-not-to-hoist}).  A simple example of hoisting in action
is shown on \cref{fig:simple-hoisting}.

\begin{figure}
\begin{center}
\begin{bcolorcode}
map(fn int (int x) =>
      let k = y + z in
      x + k,
    a)
\end{bcolorcode}
\hspace{1cm}
\adjustbox{valign=t}{
\(\Rightarrow\)
}%
\hspace{1cm}
\begin{bcolorcode}
let k = y + z in
map(fn int (int x) =>
      x + k,
    a)
\end{bcolorcode}
\end{center}

\caption{Hoisting in action}
\label{fig:simple-hoisting}
\end{figure}

At first glance, hoisting may seem to apply too rarely to be of much
benefit, since most programmers would put loop-invariant code outside
of the loop in the first place.  However, there are two important use
cases that do not involve programmer-written code:

\begin{itemize}
\item Much \LO{} code is not written by the programmer, but is rather
  the result of program transformation by the compiler.  Inlining and
  constant folding may easily result in the creation of loop-invariant
  expressions within a loop.

\item Explicit bounds checks, as introduced in \cref{sec:assertions},
  can sometimes be hoisted out of inner loops.
\end{itemize}

The latter case merits futher elaboration.  Consider the following
program:

\begin{colorcode}
map(fn int (int i) =>
      a[i] + a[i*2],
    iota(n))
\end{colorcode}

Here, \texttt{a} is a free variable.  Once the compiler has turned the
implicit bound checks explicit, the program will look like this:

\begin{colorcode}
map(fn int (int i) =>
      let c1 = assert(i   >= 0 && i   < size(0,a)) in
      let c2 = assert(i*2 >= 0 && i*2 < size(0,a)) in
      a[<c1>|i] + a[<c2>|i*2],
    iota(n))
\end{colorcode}

Now, the assertions are not loop-invariant, as they depend on
\texttt{i}, but using \textit{symbolic range
  propagation}\cite{blume1995symbolic}, it can be deduced that the
variable \textit{i} will always be in the range $[0,n-1]$.  This
allows us to rewrite the assertions -- the checks for non-negativity
goes away, as it is always true, and we only have to check the upper
bound for the maximum values that \texttt{i} and \texttt{i*2} may
attain:

\begin{colorcode}
map(fn int (int i) =>
      let c1 = assert(n   < size(0,a)) in
      let c2 = assert(n*2 < size(0,a)) in
      a[<c1>|i] + a[<c2>|i*2],
    iota(n))
\end{colorcode}

Now \texttt{c1} and \texttt{c2} are loop-invariant, and we can move
them out of the loop body, and perform bounds checking just once,
before entering the loop:

\begin{colorcode}
let c1 = assert(n   < size(0,a)) in
let c2 = assert(n*2 < size(0,a)) in
map(fn int (int i) =>
      a[<c1>|i] + a[<c2>|i*2],
    iota(n))
\end{colorcode}

For a simple loop such as the above, the potential benefits are great,
as most of the instructions of the original loop body was devoted to
bounds checkings.

The \LO{} compiler does not yet support the symbolic range propagation
that enables the critical rewrite of the \texttt{assert} expressions.
An unpublished bachelors thesis by Jonas Brunsgaard and Rasmus Wriedt
Larsen has demonstrated that the technique works in practice, but
their work has not yet been merged with the main compiler code base.

Hoisting assertions such as these is not only useful when the program
uses explicit array indexing.  While the array accesses performed by
SOACs are by construction always in-bounds, and therefore do not need
dynamic checks, the \texttt{assert} expressions we generate when
transforming from external SOACs to tupleless SOACs are conceptually
identical to bounds checks, and similarl important to hoist.  For
example, consider this program:

\begin{colorcode}
map(fn [int] ([int] r) =>
      map(op+, zip(r, b)),
    a)
\end{colorcode}

After transformation to internal \LO{}, we get the following:

\begin{colorcode}
mapT(fn \{[int]\} ([int] r) =>
       let c = assert(size(0,r) = size(0,b)) in
       <c>mapT(op+, r, b),
    a)
\end{colorcode}

The assertion is not loop-invariant, and range analysis is no help.
For this case, structural size analysis (described in depth in
\cref{sec:size-analysis}) reveals that since \texttt{r} is a row of
\texttt{a}, the outer size of \texttt{r} (\texttt{size(0,r)}) is equal
to the inner size of \texttt{a} (\texttt{size(1,a)}).  Thus, the
compiler rewrites to:

\begin{colorcode}
mapT(fn \{[int]\} ([int] r) =>
       let c = assert(size(1,a) = size(0,b)) in
       <c>mapT(op+, r, b),
    a)
\end{colorcode}

The assertion is now loop-invariant and can be hoisted:

\begin{colorcode}
let c = assert(size(1,a) = size(0,b)) in
mapT(fn \{[int]\} ([int] r) =>
       <c>mapT(op+, r, b),
    a)
\end{colorcode}

The details of how hoisting is implemented in the \LO{} compiler is
covered in \cref{sec:rebinder}.

\subsection{What not to hoist}
\label{sec:when-not-to-hoist}

Clearly, we can hoist only loop-invariant expressions.  Unfortunately,
not all loop-invariant expressions are hoistable, and as usual, when
seemingly valid transformations become problematic, constraints
imposed by uniqueness types are at fault.  Consider the following
program:

\begin{colorcode}
map(fn (int i) =>
      let a = iota(10) in
      f(a, i),
    b)
\end{colorcode}

It seems that we should be able to hoist \texttt{a} out of the loop
like this:

\begin{colorcode}
let a = iota(10) in
map(fn (int i) =>
      f(a, i),
    b)
\end{colorcode}

However, if the function call \texttt{f(a,i)} consumes the \texttt{a}
argument, hoisting would result in an invalid program, as \texttt{a}
would be consumed multiple times.  We need a ``freshly allocated''
\texttt{a} for each iteration of the loop.  Hence, we must not hoist a
binding out of a loop in which it is consumed.

\subsubsection{Hoisting out of branches}

Most compilers generally do not hoist out of branches, as branching is
often used to prevent expensive execution of expensive expressions
whose value is not needed.  On many GPUs however, execution happens in
lock-step across all processors, both sides of the branch are always
executed~\cite{reducing-branch-divergence}.  This implies that
hoisting out of a branch does not cause more instructions to be
executed, and hoisting might expose the possibility of other
optimisations, such as common subexpression elimination (see
\cref{sec:cse}).

We should still be careful when hoisting expressions out of the
branches of a conditional, as it is possible that the expression may
result in an error unless the condition checked by the conditional is
true.  For example, consider this expression:

\begin{colorcode}
if x != 0 then y / x
          else 0
\end{colorcode}

If \texttt{y / x} was hoisted out of the branch, the resulting program
might end up dividing by zero.  Assertions and array indexing are
other operations that it is not safe to hoist out of a branch.  Most
expressions are safe however, and the \LO{} compiler aggressively
hoists these out of branches.  Depending on improvements in hardware,
or the targeting of \LO{} for non-GPU systems, it is likely that this
strategy will need to be revised.

\subsubsection{Performance Considerations}

There is another potential case where hoisting, while not resulting in
an invalid program, proves detrimental rather than beneficial to
performance.  This occurs when retrieving the hoisted value from
memory would be more expensive than re-computing it for each loop
iteration, which is particularly likely to occur on GPUs, as global
memory accesses are enormously expensive.  Balancing this problem is
not currently tackled by the \LO{} compiler, which instead hoists as
aggressively as possible.  Note that this is not a problem when
hoisting assertions, such as bounds checks, as the resulting values
are not actually accessed from within the loop.

\section{Common Subexpression Elimination}
\label{sec:cse}

Common Subexpression Elimination (henceforth referred to as CSE) is a
popular compiler optimisation that identifies identical expressions
(i.e. expressions that always evaluate to the same value), and
replaces them with a variable holding the value.  For example, this
program:

\begin{colorcode}
2 * x + 2 * x
\end{colorcode}

Can be transformed through CSE into:

\begin{colorcode}
let tmp = 2 * x in
tmp + tmp
\end{colorcode}

This saves us a multiplication.  As with hoisting, CSE can potentially
be detrimental to performance if it increases memory pressure, but
again like hoisting, this is not something the \LO{} compiler
currently takes into consideration.

We must be careful not to perform CSE such that we end with a
violation of the uniqueness rules.  Consider this program:

\begin{colorcode}
let a = iota(10) in
let b = iota(10) in
let a[2] = 5 in
f(a,b)
\end{colorcode}

Since \texttt{iota(10)} appears in two places, we might be tempted to
factor it out:

\begin{colorcode}
let tmp = iota(10) in
let a = tmp in
let b = tmp in
let a[2] = 5 in
f(a,b)
\end{colorcode}

However, this violates Uniqueness Rule 1, as \texttt{b} is aliased to
\texttt{a}, yet is used after \texttt{a} is consumed in a
\texttt{let-with} expression.  The solution is to not perform CSE on
expressions whose result is eventually consumed -- or more
conservatively, never perform CSE on an expression of type
\texttt{*[$\alpha$]}.  The latter is easier to implement, although too
conservative, but is what the \LO{} compiler currently does.

\section{The Rebinder}
\label{sec:rebinder}

Conceptually, hoisting and CSE are rather different transformations.
However, they both depend on identifying subexpressions that can be
moved (in the case of hoisting) or replaced (in the case of CSE), but
where the actual expression rewriting is quite simple.  In the \LO{}
compiler, the observation was made that a lot of the machinery used to
implement hoisting could be easily extended to also perform CSE, and
thus was born an optimisation with the rather idiosyncratic name
\textit{the Rebinder}.

The central idea is to assume a fully normalised program in a slightly
modified A-normal form~\cite{Sabry:1992:RPC:141478.141563}, a format
similar to \textit{three address code}, where the definition of each
\texttt{let}-binding is a \textit{simple expression}, and the body of
a \texttt{let}-binding is either another binding or a variable.  A
simple expression is either a branch\footnote{Not permitted in
  ``standard'' A-normal form.}, or an expression where all
subexpressions are variables or constants (except for the bodies of
SOAC functions), which implies that their execution terminates
immediately.  For example, the following program:

\begin{colorcode}
fun real solve(real a, real b, real c) =
  (-b + sqrt(b*b - 4.0*a*c)) / (2.0*a)
\end{colorcode}

Would look like this in A-normal form.  Note that it is also
\texttt{let}-normalised (\cref{sec:let-normalisation}):

\begin{colorcode}
fun real solve(real a_0, real b_1, real c_2) =
  let negate_3  = -b_1 in
  let times_4   = b_1 * b_1 in
  let times_5   = 4.0 * a_0 in
  let times_6   = times_5 * c_2 in
  let minus_7   = times_4 - times_6 in
  let norm_8    = sqrt(minus_7) in
  let plus_9    = negate_3 + norm_8 in
  let times_10  = 2.0 * a_0 in
  let divide_11 = plus_9 / times_10 in
  divide_11
\end{colorcode}

This simplifies hoisting and CSE significantly, as the problem is now
reduced to moving nodes in the syntax tree (for hoisting) and
substituting definitions of \texttt{let}-bindings (for CSE).
Normalisation is performed by a separate pass before entering the
Rebinder.

In order to keep the exposition simpler, \texttt{loop} and
\texttt{let-with} will be ignored (except with respect to upholding
uniqueness constraints), and only discuss hoisting of normal
\texttt{let}-bindings.

To try to give an intuition of the Rebinder, let us consider the
following contrived program:

\begin{colorcode}
fun int main([int] a, int i, int v) =
  let {res} =
    reduceT(fn \{int\} (int sum, int x) =>
              {sum + x + v*a[i]},
            \{v*a[i]\}, a) in
  res
\end{colorcode}

The goal is to hoist the loop-invariant expression \texttt{v*a[i]} out
of the loop, and use CSE to combine it with the initial value of the
accumulator.  To this end, it is first normalised:

\begin{colorcode}
fun int main([int] a_0, int i_1, int v_2) =
  let index_10 = a_0[i_1] in
  let times_11 = v_2 * index_10 in
  let {res_5} =
    reduceT(fn {int} (int sum_3, int x_4) =>
              let plus_6 = sum_3 + x_4 in
              let index_7 = a_0[i_1] in
              let times_8 = v_2 * index_7 in
              let plus_9 = plus_6 + times_8 in
              {plus_9},
            {times_11}, a_0) in
  res_5
\end{colorcode}

The central idea behind the rebinder is to maintain the
\textit{potentially hoistable set}, a partially ordered set of
bindings.  The partial order used is $\preceq$: for two bindings
$b_{i}, b_{j}$, $b_{i} \preceq b_{j}$ if $b_{j}$ uses any variables
bound by $b_{i}$, or if $b_{i} = b_{j}$.

The Rebinder proceeds with a recursive walk down the syntax tree,
inspecting the bindings:

\begin{itemize}
\item First, we encounter the \texttt{index\_10} and
  \texttt{times\_11} bindings, and their right-hand sides are
  inspected.  Neither of these inspections yield hoistable
  subexpressions, but we remove \texttt{index\_10} and
  \texttt{index\_11} themselves and insert them into the potentially
  hoistable set.

\item Next, we encounter the \texttt{res\_5} binding and we descend
  recursively into the scope of the SOAC function body:

  \begin{itemize}
  \item We inspect the right-hand sides of \texttt{plus\_6},
    \texttt{index\_7}, \texttt{times\_8} and \texttt{plus\_9}, neither
    of which yield hoistable subexpressions.  We collect these
    bindings into a potentially hoistable set and remove them from the
    function body, leaving just the expression \texttt{\{plus\}}.

  \item We are now done inspecting the function, and we need to decide
    which of the bindings in the potentially hoistable set can in fact
    be hoisted out.  We create a dependency graph of the bindings, and
    see which of the bindings have right-hand sides that depend on the
    function parameters \texttt{sum\_3} and \texttt{x\_4}.  This
    leaves the bindings for \texttt{index\_7} and \texttt{times\_8} as
    hoistable; while \texttt{plus\_6} and \texttt{plus\_9} are
    re-inserted into the program.
  \end{itemize}

  This yields the \texttt{index\_7} and \texttt{times\_8} bindings as
  new elements in the potentially hoistable set.  We also add the
  modified \texttt{res\_5} binding itself.

\item Since there are no more bindings left, and we are at the top
  level of a function, we insert the bindings in the potentially
  hoistable set (\texttt{index\_10}, \texttt{times\_11},
  \texttt{res\_5}, \texttt{index\_7} and \texttt{times\_8}) in the
  program, yielding:

\begin{colorcode}
fun int main([int] a_0, int i_1, int v_2) =
  let index_10 = a_0[i_1] in
  let times_11 = v_2 * index_10 in
  let index_7 = a_0[i_1] in
  let times_8 = v_2 * index_7 in
  let {res_5} =
    reduceT(fn \{int\} (int sum_3, int x_4) =>
              let plus_6 = sum_3 + x_4 in
              let plus_9 = plus_6 + times_8 in
              \{plus_9\},
            \{times_11\}, a_0) in
  res_5
\end{colorcode}
\end{itemize}

We could now do a separate CSE pass over the entire program, but there
may be a more efficient strategy.  The rebinder should be able to
perform the CSE optimisation whenever we insert the non-hoistable
bindings into the syntax tree.  We will try to provide an intuition
for the approach, using the above example:

When, at the end, we have to insert bindings for \texttt{index\_10},
\texttt{times\_11}, \texttt{res\_5}, \texttt{index\_7} ordering
constraints \texttt{times\_8} must first be determined.  It is seen
that \texttt{times\_11} uses \texttt{index\_10}, \texttt{times\_8}
uses \texttt{index\_7}, and \texttt{res\_5} uses \texttt{times\_11}
and \texttt{times\_8}.  Hence, the following insertion order is chosen
(top-down):

\begin{enumerate}
\item \texttt{index\_10}
\item \texttt{times\_11}
\item \texttt{index\_7}
\item \texttt{times\_8}
\item \texttt{res\_5}.
\end{enumerate}

None of these bindings can be removed, as the names they bind may be
used in subexpressions, but their right-hand sides (RHS) can be
changed freely.  This is what is exploited to perform CSE.  The
following steps are performed:

\begin{description}
\item[\texttt{index\_10}:] Inserted unchanged, but we record its
  right-RHS, in case we end up seeing an identical expression
  later on.
\item[\texttt{times\_11}:] Likewise inserted unchanged, as its RHS
  does not correspond to any previously seen.
\item[\texttt{index\_7}:] Since the RHS of this binding is identical
  to the RHS of \texttt{index\_10}, we replace the RHS of
  \texttt{index\_7} with the variable \texttt{index\_10}.  We insert
  the resulting binding \texttt{let~times\_11~=~index\_7} and note the
  fact that this substitution has performed.
\item[\texttt{times\_8}:] The substitution
  $\texttt{index\_y}\rightarrow\texttt{index\_10}$ is performed on the
  immediate subexpressions of the RHS, obtaining
  \texttt{v\_2~*~index\_10}, then check whether the resulting
  expression has been seen before.  This turns out to be identical to
  the RHS of \texttt{times\_11}, and therefore we insert the binding
  \texttt{let~times\_8~=~times\_11}.
\item[\texttt{res\_5}:] The substitution
  $\texttt{index\_y}\rightarrow\texttt{index\_10},\texttt{times\_8}\rightarrow\texttt{times\_11}$
  is performed, although no changes are made.  The binding is then
  inserted.
\end{description}

The result is the following program:

\begin{colorcode}
fun int main([int] a_0, int i_1, int v_2) =
  let index_10 = a_0[i_1] in
  let times_11 = v_2 * index_10 in
  let index_7 = index_10 in
  let times_8 = times_11 in
  let {res_5} =
    reduceT(fn \{int\} (int sum_3, int x_4) =>
              let plus_6 = sum_3 + x_4 in
              let plus_9 = plus_6 + times_8 in
              \{plus_9\},
            \{times_11\}, a_0) in
  res_5
\end{colorcode}

Copy propagation can then be used to remove the \texttt{index\_7} and
\texttt{times\_8} bindings.\fixme{This is still way too vague and hard to understand.}

\subsection{Hoisting Bindings}

The problem is as follows: We are given a potentially hoistable set,
$\{(p_{i}, e_{i})\}$, of patterns $p_{i}$ and $e_{i}$.  Each pattern
$p_{i}$ defines several names that may be used by other expressions in
the set.

We need to split the potentially hoistable set into a
\textit{hoistable set}, and a set of the bindings that cannot be
hoisted further, the \textit{unhoistable set}.  To this end, we are
given a unary relation $\mathcal{B}$, that given a binding $(p_{i},
e_{i})$, tells us whether $e_{i}$ is \texttt{it} blocked from being
hoisted.  As the most archetypical reason, $\mathcal{B}(p_{i}, e_{i})$
whenever $e_{i}$ uses a parameter of the function that we are trying
to hoist out of, but the relation also checks the other cases
mentioned in \cref{sec:when-not-to-hoist}.

Each of the expressions $e_{i}$ may use variables from any of the
patterns $p_{j}$ (except its own), but there may also be free
variables that are not contained in any pattern in the potentially
hoistable set.  These correspond to names bound higher up in the
program.

Since the potentially hoistable set is partially ordered, we can
process its elements in dependency order.  We keep track of the names
bound by unhoistable bindings in an \textit{unhoisted name set},
initially empty.  For each binding $(p_{i},e_{i})$, we check whether
$\mathcal{B}(p_{i},e_{i})$ or $e_{i}$ uses a name in the unhoisted
name set:
\begin{itemize}
\item If so, the binding is put into the unhoistable set, and the
  names in $p_{i}$ are added to the unhoisted name set.
\item Otherwise, the binding is put into the hoistable set.
\end{itemize}
The end result: a set of hoistable bindings and a set of nonhoistable
bindings.  Because of the ordering, we are guaranteed that no binding
in the former uses a name bound bound in the latter.

\subsection{Inserting Bindings}

\fixme{Explain the ``insertion'' part.}

\subsection{Problems with normalisation}

Is (x*y)*z the same as x*(y*z)?  They normalise differently. \fixme{Finish}

%%% Local Variables: 
%%% mode: latex
%%% TeX-master: "thesis.tex"
%%% End: 
