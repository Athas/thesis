\chapter{Uniqueness types}
\label{chap:uniqueness-types}

While \LO{} is through and through a pure functional language, it may
occasionally prove useful to express certain algorithms in an
imperative style.  Consider a function for computing the $n$ first
Fibonacci numbers:

\begin{colorcode}
fun [int] fib(int n) =
  // Create "empty" array.
  let arr = iota(n) in
  // Fill array with Fibonacci numbers.
  loop (arr) = for i < n-2 do
                 let arr[i+2] = arr[i] + arr[i+1]
                 in arr
  in arr
\end{colorcode}

If the array \texttt{arr} is copied for each iteration of the loop, we
are going to put enormous pressure on memory, and spend a lot of time
moving around data, even though it is clear in this case that the
``old'' value of \texttt{arr} will never be used again.  Precisely,
what should be an algorithm with complexity $O(n)$ becomes $O(n^2)$,
due to copying the size $n$ array (an $O(n)$ operation) for each of
the $n$ iterations of the loop.

To prevent this, we want want to update the array \textit{in-place},
that is, with a static guarantee that the operation will not require
any additional memory allocation, such as copying the array.  With an
in-place modification, a \texttt{let-with} can modify the array in
time proportional to the slice being updated ($O(1)$ in the case of
the Fibonacci function), rather than time proportional to the size of
the final array, as is the case if we perform a copy.  In order to
perform the update without violating referential transparency, we need
to know that no other references to the array exists, or at least that
such references will not be used on any execution path following the
in-place update.

In \LO{}, this is done through a type system feature I call
\textit{uniqueness types}, similar to, although simpler, than the
uniqueness types of
Clean~\cite{clean-uniqueness-types,barendsen1996uniqueness}.
Alongside a (relatively) simple aliasing analysis in the type checker,
this is sufficient to determine at compile time whether an in-place
modification is safe, and signal a compile time error if
\texttt{let-with} is used in way where safety cannot be guaranteed.
This means that \texttt{let-with} must \textit{always} be efficient,
and its use is not permitted otherwise.

The simplest way to introduce uniqueness types is through examples.
To that end, let us consider the following function definition.

\begin{colorcode}
fun \emp{*}[int] modify(\emp{*}[int] a, int i, int x) =
  let b = a with [i] <- a[i] + x in
  b
\end{colorcode}

The function call \texttt{modify($a$,$i$,$x$)} returns $a$, but where
the element at index \texttt{i} has been increased by $x$.  Note the
\emp{asterisks}: in the parameter declaration \texttt{*[int] a}, this
means that the function \texttt{modify} has been given ``ownership''
of the array $a$, meaning that any caller of \texttt{modify} will
never reference array $a$ after the call again.  In particular,
\texttt{modify} can change the element at index \texttt{i} without
first copying the array, i.e. \texttt{modify} is free to do an
in-place modification.  Furthermore, the return value of
\texttt{modify} is also unique - this means that the result of the
call to \texttt{modify} does not share elements with any other visible
variables.

Let us consider a call to \texttt{modify}, which might look as
follows.

\begin{colorcode}
let \(b\) = modify(\(a\), \(i\), \(x\)) in
...
\end{colorcode}

Under which circumstances is this call valid?  Two things must hold:
\begin{enumerate}
\item The type of \texttt{$a$} must be \texttt{*[int]}, of course.

\item Neither \texttt{$a$} or any variable that \textit{aliases}
  \texttt{$a$} may be used on any execution path following the call to
  \texttt{modify}.
\end{enumerate}

In general, when a value is passed as a unique-typed argument in a
function call, we consider that value to be \textit{consumed}, and
neither it nor any of its aliases can be used again.  Otherwise, we
would break the contract that gives the function liberty to manipulate
the argument however it wants.  Note that it is the type in the
argument declaration that must be unique - it is permissible to pass a
unique-typed variable as a non-unique argument (that is, a unique type
is a subtype of the corresponding nonunique type).

A variable $v$ aliases $a$ if they may share some elements,
i.e. overlap in memory.  As the most trivial case, after evaluating
the binding \texttt{let b = a}, the variable \texttt{b} will alias
\texttt{$a$}.  As another example, if we extract a row from a
two-dimensional array, the row will alias its source:
\begin{colorcode}
let b = a[0] in
... // b is aliased to a (assuming a is not one-dimensional)
\end{colorcode}
\Cref{sec:l0-sharing} will cover sharing and sharing analysis in
greater detail.

Let us consider the definition of a function returning a unique array:

\begin{colorcode}
fun *[int] f([int] a) = \(e\)
\end{colorcode}

Note that the argument, \texttt{$a$}, is non-unique, and hence we
cannot modify it.  There is another restriction as well: \texttt{$a$}
must not be aliased to our return value, as the uniqueness contract
requires us to ensure that there are no other references to the unique
return value.  This requirement would be violated if we permitted the
return value in a unique-returning function to alias its (non-unique)
parameters.

We can crystallise these observations as three principal rules:

\begin{description}
\item[Uniqueness Rule 1] When a value is passed in the place of a
  unique parameter in a function call, or used as the source in a
  \texttt{let-with} expression, neither that value, nor any value that
  aliases it, may be used on any execution path following the function
  call.  A violation of this rule is illustrated on
  \cref{fig:uniqueness-rule-1-violation}.

\item[Uniqueness Rule 2] If a function definition is declared to
  return a unique value, the return value (that is, the result of the
  body of the function) must not share memory with any non-unique
  arguments to the function.  As a consequence, at the time of
  execution, the result of a call to the function is the only
  reference to that value.  A violation of this rule is illustrated on
  \cref{fig:uniqueness-rule-2-violation}.

\item[Uniqueness Rule 3] If a function call yields a unique return
  value, the caller has exclusive access to that value.  At
  \textit{the point the call returns}, the return value may not share
  memory with any variable used in any execution path following the
  function call.  This rule is particularly subtle, but can be
  considered a rephrasing of Uniqueness Rule 2 from the ``calling
  side''.
\end{description}

\begin{figure}
\centering
\begin{colorcode}
let b = a with [i] <- 2 in
f(b,a) // \emp{Error:} a used after being source in a let-with
\end{colorcode}
\caption{Violation of Uniqueness Rule 1}
\label{fig:uniqueness-rule-1-violation}
\end{figure}

\begin{figure}
\centering
\begin{colorcode}
fun *[int] broken([[int]] a, int i) =
  a[i] // Return value aliased with 'a'.
\end{colorcode}
\caption{Violation of Uniqueness Rule 2}
\label{fig:uniqueness-rule-2-violation}
\end{figure}

Finally, it is worth emphasising that everything in this chapter is
used as part of a static analysis.  \textit{All} violations of the
uniqueness rules will be discovered at compile time (in fact, during
type-checking), thus leaving the code generator and runtime system at
liberty to exploit them for low-level optimisation.

\section{Sharing analysis}
\label{sec:l0-sharing}

Whenever the memory regions for two values overlap, we say that they
are \textit{aliased}, or that \textit{sharing} is present.  As an
example, if you have a two-dimensional array \texttt{a} and extract
its first row as the one-dimensional array \texttt{b}, we say that
\texttt{a} and \texttt{b} are aliased.  While the \LO{} compiler may
do a deep copy if it wishes\footnote{At some point, a proper cost
  model for \LO{} will be developed, and it is very likely that we
  require such indexing to be $O(1)$.}, it is not required, and this
operation thus holds the potential for sharing memory.  Sharing
analysis is necessarily conservative, and merely imposes an upper
bound on the amount of sharing happening at runtime.  The sharing
analysis in \LO{} has been carefully designed to make the bound as
tight as possible, but still easily computable.

In \LO{}, the only values that can have any sharing are arrays -
everything else is considered ``primitive''.  Tuples are special, in
that they are not considered to have any identity beyond their
elements.  Therefore, when we store sharing information for a
tuple-typed expression, we do it for each of its element types, rather
than the tuple value as a whole.

To be precise, sharing information for an expression $e$, written
$\aliases{e}$, can take one of two forms:

\begin{enumerate}
\item $l$, where $l$ is a subset of the variables in scope at $e$.
  This means that $e$ may share data with some of the variables in
  $l$.  This is the sharing information when the type of $e$ is not a
  tuple.

\item $\langle l_{1}, \ldots, l_{n} \rangle$, which requires that the
  type of $e$ is a tuple $\{t_{1}, \ldots, t_{n}\}$, and denotes that
  the sharing of the $i$th component is $s_{i}$.  This is the shape of
  the sharing information when the type of $e$ is a tuple.
\end{enumerate}

We need a way to combine sharing information.  The typical case is
computing sharing information for the expression \texttt{if c then e1
  else e2}, where the sharing of the resulting value is the
``combination'' of the sharing in both \texttt{e1} and \texttt{e2}.
We make this combination precise by the associative, commutative
operation $s_{1} \oplus s_{2}$, which is defined by the following
equation.

\begin{align*}
  l_{1} \oplus l_{2} &= l_{1} \cup l_{2} \\
  \langle l_{1}, \ldots, l_{n} \rangle \oplus \langle l_{n+1}, \ldots, l_{2n} \rangle &= \langle l_{1} \oplus l_{n+1}, \ldots, l_{n} \oplus l_{2n} \rangle \\
\end{align*}

Now we can define
\[
\aliases{\texttt{if c then e1 else e2}} = \aliases{\texttt{e1}} \oplus \aliases{\texttt{e2}}.
\]

We will often treat sharing information as a set and write things such
as $\forall v\in\aliases{e}.p(v)$ -- in these cases, the set elements
are all variables contained anywhere in the sharing information.

Aliasing is transitive -- if $v\in\aliases{e}$ and $v'\in\aliases{e}$,
then $v\in\aliases{v'}$.  Aliasing is mostly intuitive, but here's a
few rules describing how sharing is propagated by various expressions.

\begin{align*}
  \aliases{\texttt{$e$}} &= \emptyset\ (\text{Whenever $e$ has a basic type}) \\
  \aliases{\texttt{$v$}} &= \{v\} \cup \{\textrm{Any variable in scope that aliases \(v\)}\} \\
  \aliases{\texttt{$a$[i]}} &= \aliases{a} \\
  \aliases{\texttt{copy($e$)}} &= \emptyset \\
  \aliases{\texttt{if $c$ then ${e}_{1}$ else ${e}_{2}$}} &= \aliases{e_{1}} \oplus \aliases{e_{2}} \\
\end{align*}

The rule for function application is more complicated.  To begin with,
and this was indeed the original rule in \LO{}, we can state that the
return value of a function call aliases all of its arguments.

\begin{align*}
  \aliases{\textit{f}(e_{1}, \ldots, e_{n})} &= \bigcup_{1 \leq i \leq n} \aliases{e_{i}} & \textit{(--Too restrictive!)}
\end{align*}

However, it turns out that this is far too restrictive.  Consider a
call \texttt{f1($a$)} to the function \texttt{f1} whose type is shown
on \cref{fig:unique-arguments} - if the return value aliased the
argument $a$, then we could never use the return value at all, as it
would alias something that has been consumed, namely the parameter
$a$:

\begin{colorcode}
let x = f1(a) in // Now 'x' would alias 'a'.
x                // Violates Uniqueness Rule 1,
                 // as something aliasing 'a' is accessed
\end{colorcode}

Hence, a first elaboration is that the return value should only alias
those function arguments that are not consumed:

\begin{align*}
  \aliases{\textit{f}(e_{1}, \ldots, e_{n})} &= \bigcup_{1 \leq i \leq n, \text{$e_{i}$ is not consumed}} \aliases{e_{i}}  & \textit{(--Still too restrictive!)}
\end{align*}

The argument for the soundness of this rule is as follows: even if the
return value may at runtime alias a consumed argument, we do not need
to record it, as that argument will never be accessed elsewhere.

Unfortunately, the above rule is still too restrictive, as can be
illustrated by function \texttt{f2} from \cref{fig:unique-arguments}.
Consider a call \texttt{f2($a$)} - by the above rule, the return value
would be aliased to $a$, which would violate Uniqueness Rule 3, as $a$
may be used again.

Hence, we add another elaboration, wherein the alias set is empty if
the return value is unique.

\begin{align*}
  \aliases{f(e_{1}, \ldots, e_{n})} &=
  \begin{cases}
    \emptyset & \mbox{If $f$ returns an unique value}\\
    \bigcup_{\overset{1 \leq i \leq n}{\text{$e_{i}$ is not consumed}}} \aliases{e_{i}} & \mbox{Otherwise}
  \end{cases}
\end{align*}

The final rule is essentially correct, except that it ignores tuples.
As mentioned earlier, sharing information for tuples is represented
element-wise.  Hence, we can simply apply the above rule piecewise for
each element in the tuple.

\begin{figure}
\begin{center}
\begin{bcolorcode}
fun [int] f1(*[int] a) = ...

fun *[int] f2([int] a) = ...
\end{bcolorcode}
\end{center}
\caption{Unique arguments}
\label{fig:unique-arguments}
\end{figure}

Although the current aliasing rules for function calls have proven
sufficient for now, there are cases where it is too conservative.
Consider the following function.

\begin{colorcode}
fun [int] contrived([[int]] src, [int] indexes, int i) =
  src[indexes[i]]
\end{colorcode}

In a call \texttt{contrived(src,indexes,i)}, by the above rules, we
would consider the return value to be aliased to \textit{both}
\texttt{src} and \texttt{indexes}, as both are non-consumed
parameters.  Yet, it is clear by inspecting the actual function
definition that the return value will only index the \texttt{src}
parameter.

This problem is not solvable merely through refinement of the aliasing
rules - either the user must annotate each function with information
about which of the parameters may be aliased by the return value, or
the compiler could deduce it using some sharing inference algorithm.
As the latter would add a great deal of complexity, and the former
require a language change, we have postponed tackling this problem
until it becomes a problem in practice.

\section{Tracking uniqueness}
\label{subsec:l0-tracking-uniqueness}

Let us summarise:

\begin{itemize}
\item If the type of an array parameter is preceded by a single
  asterisk, it denotes that the array is unique, i.e., that it will
  never be reused outside of the current function.

\item The source operand to a \texttt{let-with} \textit{must} be
  unique.  If it is not, it is reported as a type error.
\end{itemize}

Let-with and function calls are the only places in which variable
consumption can happen.  As a first example, let us consider a
function that replaces the value at a given position in an integer
array.

\begin{colorcode}
  fun *[int] replace(*[int] arr, int i, int x) =
    let arr[i] = x in arr
\end{colorcode}

The type of this function expresses the fact that it consumes its
array argument, and also returns a unique array.  This permits
composition - \texttt{replace(replace(a, i1, x), i2, y)} is a valid
application.  Defining \texttt{replace} as
\begin{colorcode}
  fun [int] replace2(*[int] arr, int i, int x) =
    let arr[i] = x in arr
\end{colorcode}
would still be type correct (a unique array can be used anywhere a
nonunique is expected), but the composition
\texttt{replace2(replace2(a, i1, x), i2, y)} would no longer be well
typed.

Checking that uniqueness invariants are being upheld is far subtler
than normal type checking.  In particular, detailed sharing analysis
has to be performed, in order to ensure that after an array $a$ is
consumed, it becomes an error to use any value that may refer to
(parts of) the old value of the array.  Whenever we consume a variable
$a$, we mark as inaccessible all of its aliases.

\begin{colorcode}
  let b = a in               // Now \(\texttt{b}\in\aliases{\texttt{a}}\).
  let c = a with [i] <- x in // \(\forall{}v\in\aliases{\texttt{a}}\Rightarrow\textrm{Mark \(v\) as consumed.}\)
  b                          // Error, because \(\texttt{b}\in\aliases{\texttt{a}}\)!
\end{colorcode}

A key principle is that of \textit{sequence points} that lexically
checkpoint the use of variables.  As an example, assume that we are
given a function \texttt{f} of type \texttt{*[int]~->~int}.  That is,
\texttt{f} consumes an array and returns an integer.  The expression
\begin{colorcode}
  f(a) + a[i]
\end{colorcode}
is invalid because a consumption and observation of the same variable
happens within the same \textit{sequence}.  It is valid for a sequence
to contain multiple observations of the same variable, but if a
variable is consumed, that must be the only occurrence of the variable
(or any of its aliases) within the sequence.  Binding constructs
(lets, let-withs and loops) create sequence points that delimit
sequences.  If we rewrite the expression to coordinate the consumption
into its own sequence, all will be well.
\begin{colorcode}
  let c = a[i] // Since a[i] is of primitive type,
               // c does not alias a.
  in f(a) + c
\end{colorcode}

The reason for this rule is to enable simpler code generation, as any
necessary order of operations is evident in the code.  It does require
a certain amount of care when doing program transformations, as for
example expression reordering may result in an invalid program, as
shown on \cref{fig:reordering-uniqueness-violation} and discussed in
further detail in \cref{chap:fusion}.

\begin{figure}
\centering
\begin{minipage}{0.25\columnwidth}
\begin{colorcode}
let x = a[0] in
let b = a with
  [i] <- y in
x + b[1]
\end{colorcode}
\end{minipage}
\begin{minipage}{0.05\columnwidth}
$\Rightarrow$
\end{minipage}
\begin{minipage}{0.6\columnwidth}
\begin{colorcode}
let b = a with [i] <- y in
let x = a[0] in // \emp{Error}:
                // violates Uniqueness Rule 1
x + b[1]
\end{colorcode}
\end{minipage}
\caption{Expression reordering causing violation of uniqueness rules}
\label{fig:reordering-uniqueness-violation}
\end{figure}

In the previous examples, function arguments that were consumed were
all simple variables, making it easy to describe what was being
consumed.  But in general, we might have an expression
\begin{colorcode}
  replace(e, i, x)
\end{colorcode}
where \texttt{e} is some arbitrary expression.  In this case, we mark
as consumed all variables in $\aliases{\texttt{e}}$.

Constant, literal arrays are not considered unique, as the compiler
may put them in read-only memory and return the same reference every
time they are accessed.  For example, the following program is
invalid.
\begin{colorcode}
  fun [int] fibs(int i, int x) =
    let a = [1, 1, 2, 3, 5, 8, 13] in
    let a[i] = x in a
\end{colorcode}
Since \texttt{a} is not unique, its use in the let-with is a type
error.  However, we can use \texttt{copy} to create a unique duplicate
of the array.
\begin{colorcode}
  fun [int] fibs(int i, int x) =
    let a = copy([1, 1, 2, 3, 5, 8, 13]) in
    let a[i] = x in a
\end{colorcode}

If we have a function such as
\begin{colorcode}
  fun int f(*[int] a, int x) = x
\end{colorcode}
then it is not valid to curry it in such a way that we provide values
for the consumed parameters.  For example, \texttt{map(f ($a$), b)}
would be an error.  The reason for this is that \texttt{f} may be
called an arbitrary number of times during the mapping, but $a$ can
only be consumed once.

%%% Local Variables: 
%%% mode: latex
%%% TeX-master: "thesis.tex"
%%% End: 
