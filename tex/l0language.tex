\chapter{The \LO{} language}
\label{chap:l0language}

The \LO{} programming language is a purely functional, call-by-value,
mostly first-order language that permits bulk operations on arrays
using \textit{second-order array combinators} (SOACs).
\Cref{sec:fo-l0,sec:soacs} will present \LO{} through informal
walkthrough of the major language concepts, whilst a complete
reference of language constructs is given in \cref{sec:l0-reference}.

The primary idea behind \LO{} is to design a language that has enough
expressive power to conveniently express complex programs, yet is also
amenable to aggressive optimisation and parallelisation.
Unfortunately, as the expressive power of a language grows, the
difficulty of optimisation often rises likewise.  For example, we
support nested parallelism, despite the complexities of efficiently
mapping to the flat parallelism supported by hardware, as a great many
programs depend on this feature.  On the other hand, we do not support
non-regular arrays, as they complicate size analysis a great deal.  It
is also the plan to eventually present a rigorous cost model for
\LO{}, although this work has not yet been completed.

This chapter serves as a reference and basic introduction to the \LO{}
language.  \Cref{chap:uniqueness-types,chap:internal} describe more subtle
design issues.

\section{First-order \LO{}}
\label{sec:fo-l0}

\begin{figure}[bt]
\begin{tabular}{lrll}
$t$ & $::=$ & \texttt{int} & (Integers) \\
& $|$ & \texttt{real} & (Floats) \\
& $|$ & \texttt{bool} & (Booleans) \\
& $|$ & \texttt{char} & (Characters) \\
& $|$ & \texttt{\{$t_{1}$, \ldots, $t_{n}$\}} & (Tuples) \\
& $|$ & \texttt{[$t$]} & (Arrays) \\
& $|$ & \texttt{*[$t$]} & (Unique arrays) \\
\\
$k$ & $::=$ & $n$ & (Integer)\\
& $|$ & $x$ & (Decimal number) \\
& $|$ & $b$ & (Boolean)\\
& $|$ & $c$ & (Character)\\
& $|$ & \texttt{\{$v_{1},\ldots,v_{n}$\}} & (Tuple) \\
& $|$ & \texttt{[$v_{1},\ldots,v_{n}$]} & (Array) \\
\\
$p$ & $::=$ & \texttt{\textbf{id}} & (Name pattern)\\
& $|$ & \texttt{\{$p_{1}$, \ldots, $p_{n}$\}} & (Tuple pattern) \\
\end{tabular}
\caption{\LO{} syntax}
\label{fig:fo-syntax}
\end{figure}

\begin{figure}[bt]
\begin{tabular}{lrll}
$e$ & $::=$ & $k$ & (Constant)\\
& $|$ & $v$ & (Variable)\\
& $|$ & \texttt{\{$e_{1}$,\ldots,$e_{n}$\}} & (Tuple expression) \\
& $|$ & \texttt{[$e_{1}$,\ldots,$e_{n}$]} & (Array expression) \\
& $|$ & $e_{1} \odot{} e_{2}$ & (Binary operator) \\
& $|$ & \texttt{-$e$} & (Prefix minus) \\
& $|$ & \texttt{not $e$} & (Logical negation) \\
& $|$ & \texttt{if $e_{1}$ then $e_{2}$ else $e_{3}$} & (Branching) \\
& $|$ & \texttt{$v$[$e_{1}$, \ldots, $e_{n}$]} & (Indexing) \\
& $|$ & \texttt{$v$($e_{1}$, \ldots, $e_{n}$)} & (Function call) \\
& $|$ & \texttt{let $p$ = $e_{1}$ in $e_{2}$} & (Pattern binding) \\
& $|$ & \texttt{zip($e_{1}$, \ldots, $e_{n}$)} & (Zipping) \\
& $|$ & \texttt{unzip($e$)} & (Unzipping) \\
& $|$ & \texttt{iota($e$)} & (Range) \\
& $|$ & \texttt{replicate($e_{n}$, $e_{v}$)} & (Replication) \\
& $|$ & \texttt{size($e$)} & (Array length) \\
& $|$ & \texttt{reshape(($e_{1}$,\ldots,$e_{n}$), $e$)} & (Array reshape) \\
& $|$ & \texttt{transpose($e$)} & (Transposition) \\
& $|$ & \texttt{split($e_{1}$, $e_{2}$)} & (Split $e_{2}$ at index $e_{1}$) \\
& $|$ & \texttt{concat($e_{1}$, $e_{2}$)} & (Concatenation) \\
& $|$ & \texttt{let $v_{1}$ = $v_{2}$ with} & (In-place update) \\
&     & \texttt{\ \ [$e_{1}$,\ldots,$e_{n}$] <- $e_{v}$} \\
&     & \texttt{in $e_{b}$} \\
& $|$ & \texttt{loop ($p$ = $e_{1}$) =} & (Loop) \\
&     & \texttt{\ \ for $v$ < $e_{2}$ do $e_{3}$} \\
&     & \texttt{in $e_{4}$} \\
\end{tabular}
\\
\begin{tabular}{lrll}
$fun$ & $::=$ & \texttt{fun $t$ $v$($t_{1}$ $v_{1}$,\ldots $t_{n}$ $v_{n}$) = $e$} \\
\\
$prog$ & $::=$ & $\epsilon$ \\
       & $|$   & $fun$ $prog$
\end{tabular}
\caption{\LO{} syntax, continued}
\label{fig:fo-syntax-continued}
\end{figure}

The syntax of \LO{}, as seen on \cref{fig:fo-syntax} and
\cref{fig:fo-syntax-continued}, is heavily inspired by Haskell and
Standard ML.  An identifier starts with a letter, followed by any
number of letters, digits and underscores.  Numeric, string and
character literals use the same notation as Haskell (which is very
similar to C), including all escape characters.  Comments are
indicated with \texttt{//} and span to end of line.

An \LO{} program consists of a sequence of \emph{function
  definitions}, of the following form.

\begin{colorcode}
  fun \textit{return-type} \textit{name}(\textit{params...}) = \textit{body}
\end{colorcode}

A function must declare both its return type and the types of all its
parameters.  All functions (except for inline anonymous functions; see
below) are defined globally.  \LO{} does not use type inference.
Symbolic constants are not supported, although 0-ary functions can be
defined.  As a concrete example, here is the recursive definition of
the factorial function in \LO{}.
\begin{colorcode}
  fun int fact(int n) =
    if n = 0 then 1
             else n * fact(n-1)
\end{colorcode}
Indentation has no syntactical significance in \LO{}, but recommended for
readability.

The syntax for tuple types is a comma-separated list of types or
values enclosed in braces, so \texttt{\{int, real\}} is a pair of an
integer and a floating-point number.  Both single-element and empty
tuples are permitted.  Array types are written as the element type
surrounded by brackets, meaning that \texttt{[int]} is a
one-dimensional array of integers, and \texttt{[[[\{int, real\}]]]} is a
three-dimensional array of tuples of integers and floats.  An
immediate array is written as a sequence of elements enclosed by
brackets.
\begin{colorcode}
  [1, 2, 3]       // Array of type [int].
  [[1], [2], [3]] // Array of type [[int]].
\end{colorcode}
All arrays must be \emph{regular} (often termed \emph{full}) - for
example, all rows of a two-dimensional array must have the same number
of elements.
\begin{colorcode}
  [[1, 2], [3]]      // Compile-time error.
  [iota(1), iota(2)] // A run-time error if reached.
\end{colorcode}
The restriction to regular arrays simplifies size analysis and
optimisation.

Arrays are indexed using the common row-major notation, e.g.,
\texttt{a[i1, i2, i3...]}.  An indexing is said to be \textit{full} if
the number of given indexes is equal to the dimensionality of the
array. 

A \texttt{let}-expression can be used to refer to the result of a
subexpression:
\begin{colorcode}
  let z = x + y in ...
\end{colorcode}
Recall that \LO{} is eagerly evaluated, so the right-hand side of the
\texttt{let} is evaluated exactly once, at the time it is first
encountered.

Two-way \texttt{if-then-else} is the only branching construct in \LO{}.
Pattern matching is supported in a limited way for taking apart
tuples, but this can only be done in \texttt{let}-bindings, and not
directly in a function argument list.  Specifically, the following
function definition is not valid.
\begin{colorcode}
  fun int sumpair(\{int, int\} \{x, y\}) = x + y // WRONG!
\end{colorcode}
Instead, we must use a let-binding explicitly, as follows.
\begin{colorcode}
  fun int sumpair(\{int, int\} t) =
    let \{x,y\} = t in x + y
\end{colorcode}
Pattern-matching in a binding is the only way to access the components
of a tuple.

Function calls are written as the function name followed by the
arguments enclosed in parentheses.  All function calls must be fully
saturated - currying is only permitted in SOACs (see
\cref{sec:soacs}).

\subsection{Sequential loops}
\label{sec:sequential-loops}

\LO{} has a built-in syntax for expressing certain tail-recursive
functions.  Consider the following tail-recursive formulation of a
function for computing the Fibonacci numbers.
\begin{colorcode}
  fun int fib(int n) = fibhelper(1,1,n)

  fun int fibhelper(int x, int y, int n) =
    if n = 1 then x else fibhelper(y, x+y, n-1)
\end{colorcode}
We can rewrite this using the \texttt{loop} construct.
\begin{colorcode}
  fun int fib(int n) =
    loop (\{x, y\} = \{1,1\}) = for i < n do
                              \{y, x+y\}
    in x
\end{colorcode}
The semantics of this is precisely as in the tail-recursive function
formulation.  In general, a loop
\begin{colorcode}
  loop (\emph{pat} = \emph{initial}) = for \emph{i} < \emph{bound} do \emph{loopbody}
  in \emph{body}
\end{colorcode}
has the following semantics:

\begin{enumerate}
  \item Bind \textit{pat} to the initial values given in \textit{initial}.
  \item While $\textit{i} < \textit{bound}$, evaluate
    \textit{loopbody}, rebinding \textit{pat} to be the value returned
    by the body.  At the end of each iteration, increment \textit{i}
    by one.
  \item Evaluate \textit{body} with \textit{pat} bound to its final
    value.
\end{enumerate}
Semantically, a \texttt{loop} expression is completely equivalent to a
call to its corresponding tail-recursive function.  For example,
denoting by \texttt{t} the type of \texttt{x}, the loop in
\cref{fig:loop-recursion} has the semantics of a call to the
tail-recursive function on the right-hand side.

The purpose of \texttt{loop} is partly to render some sequential
computations slightly more convenient, but primarily to express
certain very specific forms of recursive functions, specifically those
with a fixed iteration count.  This property can eventually be used
for analysis and optimisation, although the current \LO{} compiler
does not yet exploit this.

\begin{figure}
\begin{center}
\begin{minipage}{0.3\columnwidth}
\begin{colorcode}
loop (x = a) =
  for i < n do
    g(x)
in body
\end{colorcode}
\end{minipage}
\begin{minipage}{0.05\columnwidth}
$\Rightarrow$
\end{minipage}
\begin{minipage}{0.4\columnwidth}
\begin{colorcode}
fun t f(int i, int n, t x) =
  if i >= n then x
     else f(i+1, n, g(x))

let x = f(i, n, a)
in body
\end{colorcode}
\end{minipage}
\end{center}
\caption{Equivalence between loops and recursive functions}
\label{fig:loop-recursion}
\end{figure}

\subsection{In-place updates}
\label{sec:in-place-updates}

In an array-oriented programming language, a common task is to modify
some elements of an array.  In a pure language, we cannot permit free
mutation, but we can permit the creation of a duplicate array, where
some elements have been changed.  General modification of array
elements is done using the \texttt{let-with} construct.  In its most
general form, it looks as follows.
\begin{colorcode}
  let \textit{dest} = \textit{src} with [\textit{indexes}] <- \textit{value}
  in \textit{body}
\end{colorcode}
This evaluates \textit{body} with \textit{dest} bound to the value of
\textit{src}, except that the element(s) at the position given by
\textit{indexes} take on the new value \textit{value}.\footnote{Yes,
  this is the \emph{third} binding construct in the language, ignoring
  function abstraction!}  The given indexes need not be complete, but
in that case, \textit{value} must be an array of the proper size.  As
an example, here's how we could replace the third row of an $n\times3$
array.
\begin{colorcode}
  let b = a with [2] <- [1,2,3] in b
\end{colorcode}
Whenever $\textit{dest} = \textit{src}$, we can write
\begin{colorcode}
  let \textit{dest}[\textit{indexes}] = \textit{value} in \textit{body}
\end{colorcode}
as a shortcut.  Note that this has no special semantic meaning, but is
simply a case of normal name shadowing.

For example, the loop given below implements the ``imperative''
version of matrix multiplication of two $N\times N$ matrices.

\begin{colorcode}
fun *[[int]] matmultImp(int N, [[int]] a, [[int]] b) =
    let res = replicate(N, iota(N)) in
    loop (res) = for i < N do
        loop (res) = for j < N do
            let partsum =
                let res = 0 in
                loop (res) = for k < N do
                    let res = res + a[i,k] * b[k,j]
                    in  res
                in res
            in let res[i,j] = partsum in res
        in res
    in res
\end{colorcode}

With the naive implementation based on copying the source array,
executing the \texttt{let-with} expression would require memory
proportional to the entire source array, rather than proportional to
the slice we are changing.  This is not ideal.  Therefore, the
\texttt{let-with} construct has some unusual restrictions to permit
in-place modification of the \textit{src} array, as described in
\cref{chap:uniqueness-types}.  Simply put, we track that \textit{src}
is never used again.  The consequence is that we can guarantee that
the execution of a \texttt{let-with} expression does not involve any
copying of the source array in order to create the newly bound array,
and therefore the time required for the update is proportional to the
section of the array we are updating, not the entire array.  We can
think of this as similar to array modification in an imperative
language.

\section{SOACs}
\label{sec:soacs}

The language presented in the previous section is in some sense
``sufficient'', in that it is Turing-complete, and can express
imperative-style loops in a natural way with \texttt{do}-loops.
However, \LO{} is not intended to be used in such a way - bulk
operations on arrays should be expressed via the four
\textit{second-order array combinators} (SOACs) shown in
\cref{fig:soacs}, as the optimisations covered in later chapters are
expressed as transformations on these.

\begin{figure}[bt]
\begin{tabular}{lrll}
$l$ & $::=$ & \texttt{fn $t$ ($t_{1}$ $v_{1}$, \ldots, $t_{n}$ $v_{n}$) => $e$} & (Anonymous function) \\
& $|$ & \texttt{\textbf{id} ($e_{1}$, \ldots, $e_{n}$)} & (Curried function) \\
& $|$ & \texttt{op $\odot$ ($e_{1}$, \ldots, $e_{n}$)} & (Curried operator) \\
\\
$e$ & $::=$ & \texttt{map($l$, $e$)} \\
    & $|$ & \texttt{filter($l$, $e$)} \\
    & $|$ & \texttt{reduce($l$, $x$, $e$)} \\
    & $|$ & \texttt{scan($l$, $x$, $e$)} \\
    & $|$ & \texttt{redomap($l_{r}$, $l_{m}$, $x$, $e$)} \\
\end{tabular}
\caption{Second-order array combinators}
\label{fig:soacs}
\end{figure}

The semantics of the \textsc{soac}s is identical to the similarly-named
higher-order functions found in many functional languages, but we
reproduce it here for completeness.  The types given are not \LO{}
types, but a Haskell-inspired notation, since the \textsc{soac}s cannot
be typed in \LO{} itself.
\begin{align*}
\texttt{map($f$,$a$)}
& :: (\alpha\rightarrow\beta)\rightarrow[\alpha]\rightarrow[\beta] \\
& \equiv \texttt{\{$f$($a$[0]), \ldots, $f$($a$[n])\}}
\end{align*}
\begin{align*}
\texttt{ filter}
& :: (\alpha\rightarrow\texttt{bool})\rightarrow[\alpha]\rightarrow[\alpha] \\
\texttt{ filter($f$,$a$)} & \equiv \texttt{\{$a$[i] | $f$($a$[i]) = \texttt{True }\}}
\end{align*}
\begin{align*}
\texttt{ reduce}
& :: (\alpha\rightarrow\alpha\rightarrow\alpha)\rightarrow\alpha\rightarrow[\alpha]\rightarrow\alpha \\
\texttt{ reduce($f$,$x$,$a$)} & \equiv \texttt{ $f$(\ldots($f$($f$($x$,$a$[0]), $a$[1])\ldots), $a$[n])}
\end{align*}
\begin{align*}
\texttt{ scan}
& :: (\alpha\rightarrow\alpha\rightarrow\alpha)\rightarrow\alpha\rightarrow[\alpha]\rightarrow[\alpha] \\
\texttt{ scan($f$,$x$,$a$)} & \equiv \texttt{\{$f$($x$,$a$[0]), $f$($f$($x$,$a$[0]),$a$[1]),\ldots\}}
\end{align*}

\texttt{scan} is an inclusive prefix scan, and returns an array of the
same outer size as the original array.  The functions given to
\texttt{reduce} and \texttt{scan} must be binary associative
operators, and the value given as the initial value of the accumulator
must be the neutral element for the function.  These properties are
not checked by the \LO{} compiler, and are the responsibility of the
programmer.

\texttt{redomap} is a special case -- it is not intended for use by
the programmer, but used internally for fusing \texttt{reduce} and
\texttt{map}.  Its semantics is as follows.
\begin{align*}
\texttt{redomap}
& :: (\alpha\rightarrow\alpha\rightarrow\alpha)\rightarrow(\alpha\rightarrow\beta\rightarrow\alpha)\\
& \quad\rightarrow\alpha\rightarrow[\beta]\rightarrow\alpha \\
\texttt{redomap($\odot$,$g$,$x$,$v$)} & \equiv \texttt{foldl($g$, $x$, $v$)}
\end{align*}
Note that the runtime semantics is a left-fold, not a normal \LO{}
\texttt{reduce}.  In particular, $g$ need not be associative.  We use
a Haskell-like syntax to explain the
rationale behind \texttt{redomap}:\\
$(\texttt{reduce}\ \odot\ e)\ \circ\ (\texttt{map}\ f)$ can be
formally transformed, via the list homomorphism (\textsc{lh})
promotion lemma~\cite{BirdListTh}, to an
equivalent form:  \\
$(\texttt{reduce}\ \odot\ e)\ \circ\ (\texttt{map}\ f)\
\emphh{\equiv}\ \texttt{reduce}\ \odot\
e\ \circ\ \texttt{map}\ (\emp{\texttt{reduce}\ \odot\ e\ \circ\ \texttt{map}\ f})\ \circ\ \texttt{split}_{p}$\\
where the original list is distributed to $p$ parallel processors,
each of which execute the original map-reduce computation sequentially
and, at the end, reduce in parallel the per-processor result using the
operator $\odot$.  Hence, the \textit{inner} map-reduce can be
rewritten as a left-fold:\\
$(\texttt{reduce}\ \odot\ e)\ \circ\ (\texttt{map}\ f)\
\emphh{\equiv}\ \texttt{reduce}\ \odot\
e\ \circ \texttt{map} \ (\emp{\texttt{foldl}\ g\ e})\ \circ\ \texttt{split}_{p}$\\
Where $g$ is a function generated from the composition of $f$ and
$\odot$.  It follows that in order to generate parallel code for \\
\texttt{(reduce $\odot$ $e$) $\circ$ (map $f$)} it is sufficient to
record either $\odot$ and $f$, or $\odot$ and $g$. We choose the
latter, i.e., \texttt{redomap($\odot$, $g$, $e$)}, because it allows a
richer compositional algebra for fusion.  In particular, it allows us
to fuse \texttt{reduce~$\circ$~map~$\circ$~filter} into a
\texttt{redomap} without duplicating computation, as described in
\cref{chap:fusion}.

\section{Language reference}
\label{sec:l0-reference}

The builtin types in \LO{} are \intt{}, \realt{}, \boolt{} and \chart{}, as
well as their combination in tuples and arrays.

The following list describes every syntactical language construct in
the language.

\begin{description}
  \item[\textit{constant}]\hfill\\
    Evaluates to itself.

  \item[\textit{var}]\hfill\\
    Evaluates to its value in the environment.

  \item[\texttt{\textit{x} \textbf{arithop} \textit{y}}] \hfill\\
    Evaluate the binary operator on its operands, which must both be
    of either type \intt{} or \realt.  The following operators are
    supported: \texttt{+}, \texttt{*}, \texttt{-}, \texttt{/},
    \texttt{\%}, \texttt{=}, \texttt{<}, \texttt{<=}, \texttt{pow}.

  \item[\texttt{\textit{x} \textbf{bitop} \textit{y}}] \hfill\\
    Evaluate the binary operator on its operands, which must both be
    of type \intt.  The following operators are supported:
    \texttt{\^}, \texttt{\&}, \texttt{|}, \texttt{$>>$}, \texttt{$<<$},
    i.e., bitwise xor, and, or, and arithmetic shift right and left.

  \item[\texttt{\textit{x} \&\& \textit{y}}]\hfill\\
    Logical conjunction; both operands must be of type \boolt.  Not
    short-circuiting, as this complicates program transformation.  If
    short-circuiting behaviour is desired, the programmer can use
    \texttt{if} explicitly.

  \item[\texttt{\textit{x} || \textit{y}}]\hfill\\
    Logical disjunction; both operands must be of type \boolt.  As
    with \texttt{\&\&}, not short-circuiting.

  \item[\texttt{not \textit{x}}]\hfill\\
    Logical negation of \textit{x}, which must be of type \boolt.

  \item[\texttt{-\textit{x}}]\hfill\\
    Numerical negation of \textit{x}, which must be of type \realt{} or \intt.

  \item[\texttt{a[i]}]\hfill\\
    Return the element at the given position in the array.  The index
    may be a comma-separated list of indexes.

  \item[\texttt{zip($a_1, \ldots, a_n$)}]\hfill\\
    Zips together the elements of the outer dimensions of arrays $a_1,
    \ldots, a_n$.  Static or runtime check is required to check that
    the sizes of the outermost dimension of arrays $a_1, \ldots, a_n$
    are the same.  If this invariant does not hold, program execution
    stops with an error.

  \item[\texttt{unzip($a$)}]\hfill\\
    If the type of $a$ is $[\{t_1, \ldots, t_n\}]$, the result is a
    tuple of $n$ arrays, i.e., $\{[t_1], \ldots, [t_n]\}$, otherwise
    it is a type error.

  \item[\texttt{iota(\textit{n})}]\hfill\\
    An array of the integers from $0$ to \textit{n}.

  \item[\texttt{replicate(\textit{n}, \textit{a})}]\hfill\\
    An array consisting of \textit{n} copies of \textit{a}.

  \item[\texttt{size($k$, \textit{a})}]\hfill\\
    The size of dimension $k$ of array \textit{a}.  $k$ must be a
    static integral constant.

  \item[\texttt{split(\textit{n}, \textit{a})}]\hfill\\
    Partitions the given array into two disjoint arrays
    \texttt{\textit{a}[$0\ldots{}n$]},
    \texttt{\textit{a}[$n+1\ldots{}$]}, returned as a tuple.

  \item[\texttt{concat(\textit{a}, \textit{b})}]\hfill\\
    Concatenate the rows/elements of one array with another.  The
    shape of the two arrays must be identical in all but the first
    dimension.

  \item[\texttt{copy(\textit{x})}]\hfill\\
    Return a deep copy of the argument.  Semantically, this is just
    the identity function, but it has special semantics related to
    uniqueness types as described in \cref{chap:uniqueness-types}.

  \item[\texttt{reshape((\textit{dim}$_{1}$, \ldots, \textit{dim}$_{n}$), a)}]\hfill\\
    Reshape the elements of the given array into the specified shape.
    The number of elements in \textit{a} must be equal to
    $\texttt{dim}_{1}\times\ldots\times\texttt{dim}_{n}$.

  \item[\texttt{transpose(a)}]\hfill\\
    Return the transpose of \textit{a}.

  \item[\texttt{transpose(k,n,a)}]\hfill\\
    Return the generalised transpose of \textit{a}.  If
    \texttt{$b$=transpose($k$,$n$,$a$)}, then
    \[
    \texttt{a}[i_1, \ldots, i_k, i_{k+1}, \ldots, i_{k+n}, \ldots, i_q ]
    =
    \texttt{b}[i_1 , \ldots, i_{k+1} , \ldots, i_{k+n}, i_k, \ldots, i_q ].
    \]

    Thus, \texttt{transpose(0,1,a)} is the common two-dimensional
    transpose.

    Note that $k$ and $n$ must be static integer literals, and $k+n$
    must be non-negative and smaller than the rank of \texttt{a}, or
    it is a type error.

  \item[\texttt{let \textit{pat} = \textit{e} in \textit{body}}]\hfill\\
    While evaluating \textit{body}, bind the names mentioned in
    \textit{pat} to the components in the corresponding positions of
    the value of \textit{e}.  We will refer to the expression
    \textit{e} as the ``right-hand side'' (or RHS).

  \item[\texttt{let \textit{dest} = \textit{src} with [\textit{index}] <- \textit{v} in \textit{body}}] \hfill \\
    Evaluate \textit{body} with \textit{dest} bound to the value of
    \textit{src}, except that the element(s) at the position given by
    the index take on the value of \textit{v}.  The given index need
    not be complete, but in that case, the value of \textit{v} must be
    an array of the proper size.

  \item[\texttt{if \textit{c} then \textit{a} else \textit{b}}]\hfill\\
    If \textit{c} evaluates to \textit{True}, evaluate \textit{a},
    else evaluate \textit{b}.

  \item[\texttt{loop (\textit{pat} = \textit{initial}) = for \textit{i} < \textit{bound} do \textit{loopbody} in \textit{body}}]\hfill
    \begin{enumerate}
    \item Bind \textit{pat} to the initial values given in \textit{initial}.
    \item While $\textit{i} < \textit{bound}$, evaluate \textit{loopbody},
      rebinding \textit{pat} to be the value returned by the body.
    \item Evaluate \textit{body} with \textit{pat} bound to its final
      value.
    \end{enumerate}

  \item[\texttt{map(\textit{f}, \textit{a})}]\hfill\\
    Apply \textit{f} to every element of \textit{a} and return the resulting array.

  \item[\texttt{reduce(\textit{f}, \textit{x}, \textit{a})}]\hfill\\
    Left-reduction with \textit{f} across the elements of \textit{a},
    with \textit{x} as the neutral element for \textit{f}.  \textit{f}
    must be associative, as the evaluation order is not otherwise
    specified.

  \item[\texttt{scan(\textit{f}, \textit{x}, \textit{a})}]\hfill\\
    Inclusive prefix-scan.

  \item[\texttt{filter(\textit{f}, \textit{a})}]\hfill\\
    Remove all those elements of \textit{a} that do not satisfy the
    predicate \textit{f}.

\end{description}

\subsection{Tuple shimming}

In a SOAC, if the given function expects $n$ arguments of types
$t_{1}, \ldots, t_{n}$, but the SOAC will call the function with a
single argument of type \texttt{\{$t_{1}, \ldots, t_{n}$\}} (that is,
a tuple), the \LO{} compiler will automatically generate an anonymous
unwrapping function.  This allows the following expression to
type-check (and run):

\begin{colorcode}
  map(op +, zip(as, bs))
\end{colorcode}

Without the tuple shimming, the above would cause an error, as
\texttt{op +} is a function that takes two arguments, but is passed a
two-element tuple by \texttt{map}.

\subsection{Arrays of tuples}

For reasons that will be explained in \cref{chap:fusion},
arrays of tuples are in a sense merely syntactic sugar for tuples of
arrays.  The type \texttt{[\{int, real\}]} is transformed to
\texttt{\{[int], [real]\}} during the compilation process, and all
code interacting with arrays of tuples is likewise transformed.  In
most cases, this is fully transparent to the programmer, but there are
edge cases where the transformation is not trivially an isomorphism.

Consider the type \texttt{[\{[int], [real]\}]}, which is transformed
into \texttt{\{[[int]], [[real]]\}}.  These two types are not
isomorphic, as the latter has more stringent demands as to the
fullness of arrays.  For example,
\begin{colorcode}
[
 \{[1],   [1.0]\},
 \{[2,3], [2.0]\}
]
\end{colorcode}
is a value of the former, but the first element of the
corresponding transformed tuple
\begin{colorcode}
\{
 [[1],   [2, 3]],
 [[1.0], [2.0]]
\}
\end{colorcode}
is not a full array.  Hence, when determining whether a program
generates full arrays, we must hence look at the \textit{transformed}
values - in a sense, the fullness requirement ``transcends'' the
tuples.

\Cref{sec:tuple-transformation} contains more information on the
transformation of arrays of tuples.

%%% Local Variables:
%%% mode: latex
%%% TeX-master: "thesis.tex"
%%% End:
