\chapter{Conclusions}
\label{chap:conclusions}

\section{Future work}
\label{sec:future-work}

\subsection{Size Information in Type System}

The size analysis presented in \cref{chap:hindrance-removal} is quite
restricted, and was designed and extended on an ad-hoc basis in order
to enable fusion of the real-world benchmarks.  Considering the great
importance of accurate size information in not only doing high-level
optimisations, but also generating efficient low-level code, it seems
very worthwhile to integrate tracking of array sizes into the language
itself.

I suggest a type system extension inspired by \textit{dependent
  types}, although much simpler.  As an example, let us look at how we
would like to be able to define matrix multiplication:

\begin{colorcode}
fun [[int,N],P] matMult([[int,N],M] a, [[int,M],P] b ) =
  ...
\end{colorcode}

This function declares that it takes two \texttt{int} array arguments,
the first of size $N \times M$ and the second of size $M \times P$,
and returns an integer array of size $N \times P$.  Any caller of
\texttt{matMult} must first prove to the type system that the
arguments have the correct size, while \texttt{matMult} itself must
prove that its body always returns an array of the appropriate size.

Such a proof could be provided through a mechanism much like the
current \texttt{assert}, which allows us a sort of ``escape hatch''
for when we cannot statically guarantee the size of our data - for
example, when it is given to us as input from the outside world, or
the result of a \texttt{filter}.  As the current \LO{} compiler is
already able to optimise and hoist many assertions away, this would be
useful by itself.

However, this would not solve the problem encountered in
\cref{chap:optimisation-results}, when the inability to transform a
\texttt{size} expression prevented fusion in program R0.  The
problematic part of RO has this essential structure (where \texttt{N}
is some variable in scope):

\begin{colorcode}
let b = map(fn [real] (int x) =>
              let xa = replicate(N,x) in
              loop (xa) = for i < N do
                let xa = f(xa) // Does not change size of xa
                in
              xa,
            a) in
let n = size(1,b) in
map(g(n), b)
\end{colorcode}

The current ad-hoc size analyser is not smart enough to figure out the
inner size (\texttt{N}) of the array \texttt{b}.  Integrating size
information into the type system would allow us to annotate the return
type of the anonymous function as follows:

\begin{colorcode}
let b = map(fn [real,\emp{N}] (int x) =>
              ...,
            a) in
...
\end{colorcode}

We now statically promise that the inner size of \texttt{b} will
always be \texttt{N}.  The intent is that this promise can be checked
by the type-checker.  Presumably, for the body of the function to be
type-correct, the function \texttt{f} would have been defined to
return an array of the same size as its input.

It is not yet clear exactly how we should deal with cases where the
size of an array dimension cannot be statically known, or where it is
the result of a complex expression.  It is not desirable to support
the full power of dependent types, nor to include a full theorem
prover in \LO{}, as this could make it very cumbersome to use \LO{} as
a compiler target language.  In the end, it is important to remember
that our primary motivation is to improve size tracking for the
benefit of optimisation and code generation.

\fixme{FINISH ME}

\subsection{Software Engineering}

The world already has plenty of papers and theses stuffed with long
listings of Haskell code, and I have therefore tried to shy away from
talking too much about the software architecture of the \LO{}
compiler.  While the overall code base is healthy and well-structured,
there are still several instances of technical debt that should be
paid off:

\begin{itemize}
\item While the compiler is nicely divided into discrete passes, the
  order in which said passes should be invoked is a bit unclear.  As
  it stands, programs are passed through every pass several times,
  simply to ensure that they get optimised fully.  This requires some
  bit of re-engineering, probably also involving changing some passes
  (particularly the fusion module) to be less sensitive as to the
  shape of the input program.

\item For my thesis, I have divided \LO{} into an external and
  internal language.  In the compiler, both of these are included in
  the same abstract syntax tree definition, with most passes either
  silently ignoring or loudly crashing if they encounter a construct
  that belongs to the external language.  This creates undue
  complexity, and should be resolved by splitting the language more
  clearly, even if the cost is some code duplication (for example, we
  might need separate but very similar parsers).

\item Somewhat related to the previous issue, the \LO{} syntax tree
  definition does not statically enforce normalisation.  Again,
  compiler passes either ignore them or crash when an un-normalised
  term is encountered.

\item Many optimisations depend on every variable in the program
  having a unique name.  This property is ensured by tagging each
  input name with a unique integer, then passing around a counter that
  can be used to generate fresh, globalle integers.  Unfortunately,
  many transformations (e.g. inlining) end up duplicating bits of
  code, which then have to be entirely renamed in order to preserve
  uniqueness.  Furthermore, passing the counter around is cumbersome,
  even if packaged in a state monad.  An alternative approach to
  handling name binding, based on de Bruijn
  indices~\cite{McBride:2004:FPI:1017472.1017477}, is being
  considered.
\end{itemize}

%%% Local Variables: 
%%% mode: latex
%%% TeX-master: "thesis.tex"
%%% End: 
