A pure functional data-parallel programming language, \LO{},
supporting nested parallelism on regular data, is presented.  The
language is aimed at execution on massively parallel SIMD hardware,
such as modern graphics-programming units.  A system of
\textit{uniqueness types} is included, similar to
Clean~\cite{clean-uniqueness-types}, but in our case used to permit
in-place modification of arrays where possible violating referential
transparency, which is a novel feature not found in similar
languages~\cite{chakravarty2011accelerating,keller2010regular,mcdonell2013optimising}.
Furthermore, a language-integrated assertion system is presented,
which is used by the compiler to express dynamic checks such as bounds
checks directly in \LO{}, potentially facilitating their removal from
inner loops through standard optimisations.

I discuss the design and implementation of several optimisations,
notably hoisting bounds checks out of inner loops, and loop fusion
based on a structural transformation.  The fusion transformation is
capable of fusing loops whose output is used in multiple places, when
possible without duplicating computation, which is a feature not found
in other implementations of fusion~\cite{jones2001playing}.

The benefits of my optimisations are demonstrated on three real-world
financial benchmarks.  Their translation to \LO{}, particularly their
beneficial use of in-place array modification, validates the
expressivity of the language.  It is shown that the compiler is able
to hoist bounds checks and other assertions outside of loops.

The effectiveness of fusion is demonstrated via compiler
instrumentation and quantitative and qualitative measurements on the
three benchmarks, in the form of inspecting the changes in program
dataflow.  This shows that always refusing to duplicate computation is
too conservative on parallel hardware, and discuss potential
directions for further improvement.

Finally, I report a sequential speedup as high as 71\% when comparing
the fused and optimised programs to the original version of the code.

%%% Local Variables: 
%%% mode: latex
%%% TeX-master: "thesis.tex"
%%% End: 
