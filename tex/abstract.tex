We present \LO{}, a purely functional programing language supporting nested
regular data parallelism and targeting massively parallel SIMD hardware
such as modern graphics processing units (GPUs).

\LO{} incorporates the following novel features:

\begin{itemize}
\item A type system for in-place modification and aliasing of arrays
  and array slices that ensures referential transparency, which in
  turn supports equational reasoning.

\item An assertion language for expressing bounds checks on
  dynamically allocated arrays, which can often be checked statically
  to eliminate dynamic bounds checks.

\item Compiler optimizations for hoisting bounds checks out of inner
  loops and performing loop fusion based on structural
  transformations.
\end{itemize}

We show that:

\begin{itemize}
\item The type system is simpler than existing linear and unique
  typing systems such Clean~\cite{barendsen1996uniqueness}, and more
  expressive than libraries such as DPH, Repa and
  Accelerate~\cite{Chak06DPH,keller2010regular,ArrayAccelerate}[Keller
  2010, Chakravarty 2011], for efficient array processing.

\item Our fusion transformation is capable of fusing loops whose
  output is used in multiple places, when possible without duplicating
  computation, a feature not found in other implementations of
  fusion~\cite{jones2001playing}.

\item The effectiveness of our optimizations is demonstrated on three
  real-world benchmark problems from quantitative finance, based on
  empirical run-time measurements and manual inspection of the
  optimised programs.  In particular, hoisting and fusion yield a
  sequential speed-up of up to 71\% compared to the unoptimized source
  code, even without any parallel execution.
\end{itemize}

The results suggest that the language design, expressiveness and
optimization techniques of \LO{} can be realized across a range of
SIMD-architectures, notably existing GPUs and manycore-chips, to
simultaneously achieve portability and eventually performance
competitive with hand-coding.

The results reported are based on joint work with Cosmin Oancea, DIKU.

%%% Local Variables: 
%%% mode: latex
%%% TeX-master: "thesis.tex"
%%% End: 
