\chapter{Fusion-enabling SOAC Transformations}
\label{chap:fusion-enabling-soac-transformations}

The fusion rules in \cref{sec:fusion-rules} cover only simple cases,
where the output of the producer is used directly by the consumer,
without any intermediary steps.  This means that the following
program, where the output of the producer is first passed through
\texttt{transpose}, cannot be fused.
\begin{colorcode}
let \{\emphh{b}\} = mapT(f, a) in
mapT(fn [int] ([int] r) => mapT(g, r), \emp{transpose}(\emphh{b}))
\end{colorcode}
However, it is actually possible to fuse this case by first moving the
transposition to after the consumer instead:
\begin{colorcode}
let \{\emphh{b}\} = mapT(f, a) in
\emp{transpose}(mapT(fn [int] ([int] r) => mapT(g, r), \emphh{b}))
\end{colorcode}
After this transformation, the simple \texttt{map}-\texttt{map} fusion
rule applies.  When moving around transformations such as
\texttt{transpose} (and, later, \texttt{reshape}), remember that we
think of them as having a delayed representation, and hence moving
them will not influence the cost model of the program.

In many cases, such rewriting of a producer-consumer pair is necessary
before the simple fusion rules can apply.  Indeed, one might consider
the \textsc{Fuse-Map-Redomap} rule a particularly simple example of
such a rewriting.  Fortunately, these rewriting schemes can be
incorporated into the existing fusion framework simply by considering
them as additional inference rules.  The rewriting above can be
defined as follows (using the nested map notation from
\cref{fig:nested-maps}):

\begin{align*}
  \inference{
    \fusesto
    {e_{b}}
    {\texttt{mapT($l_{a}$,$e_a$)}}
    {\texttt{mapT$^{2}$($l_{b}$,$e_b$)}}
    {soac}
  }{
    \fusesto
    {e_{b}}
    {\texttt{mapT($l_{a}$,$e_a$)}}
    {\texttt{mapT$^{2}$($l_{b}$,transpose($e_b$))}}
    {\texttt{transpose($soac$)}}
  }
  \tagsc{Fuse-Map-Transpose-Map-Single}
\end{align*}

\begin{figure}
\begin{center}
\begin{colorcode}
mapT\(\myindu{1}\)(f, a\(\myindx{1}\), \ldots , a\(\myindx{k}\)) \emphh{\(\equiv\)}
mapT(f, a\(\myindx1\), \ldots, a\(\myindx{k}\))

mapT\(\myindu{n+1}\)(f, a\(\myindx{1}\), \ldots, a\(\myindx{k}\)) \emphh{\(\equiv\)}
mapT(fn \{[\(\beta\myindx{1}\)], \ldots, [\(\beta\myindx{t}\)]\} ([\(\alpha\myindx{1}\)] x\(\myindx{1}\), \ldots, [\(\alpha\myindx{k}\)] x\(\myindx{k}\))) =>
       mapT\(\myindu{n}\)(f, x\(\myindx{1}\), \ldots, x\(\myindx{k}\))
\end{colorcode}
\end{center}
\caption{Nested map notation}
\label{fig:nested-maps}
\end{figure}

Of course, this rule is far too specific - it covers only
\texttt{map}-producers and -consumers, and with a single output and
input respectively.  In the next section, we will see a more general
treatment of when we can fuse across arrays.

Furthermore, we will need to extend the SOAC notation we use for
expressing fusion judgements.  The
\textsc{Fuse-Map-Transpose-Map-Single} rule uses \texttt{transpose} in
places where we have previously considered only plain variables and
SOAC expressions.  Their meaning is as follows:

\begin{itemize}
\item Whenever we use \texttt{transpose} (or, later, \texttt{reshape})
  around a SOAC, the intent is that we transpose every output of the
  SOAC.  Thus, even though \texttt{transpose(mapT(\ldots))} is
  technically not type-correct \LO{}, since \texttt{mapT} returns a
  tuple, the intended meaning is that every output is transposed.
\item When we enclose inputs to a SOAC in \texttt{transpose} or
  \texttt{reshape}, as in for example
  \texttt{mapT($f$,transpose($\overline{es}$))}, the intended meaning
  is to apply \texttt{transpose} to every input, i.e.
  \texttt{mapT($f$, transpose($e_{1}$), \ldots, transpose($e_{n}$))}.
\end{itemize}

Conceptually, we integrate \texttt{transpose} and \texttt{reshape}
into the consumers by inlining them in the input positions prior to
fusion.  This is illustrated on \cref{fig:inlining-transpose}

\begin{figure}
\centering
\begin{subfigure}[t]{.5\textwidth}
\begin{colorcode}
let \{b\} = mapT(f, a) in
let \{c\} = transpose(b) in
mapT(g, c)
\end{colorcode}
\subcaption{Before inlining}
\end{subfigure}%
\begin{subfigure}[t]{.5\textwidth}
\begin{colorcode}
let \{b\} = mapT(f, a) in
mapT(g, transpose(b))

\end{colorcode}
\subcaption{After inlining}
\end{subfigure}
\caption{Inlining transposition}
\label{fig:inlining-transpose}
\end{figure}

We add the following two ancillary fusion rules for fusing with a
consumer whose output is transposed or reshaped.

\begin{align*}
  \inference{
    \fusesto
    {os}
    {soac_{p}}
    {soac_{c}}
    {soac_{r}}
  }{
    \fusesto
    {os}
    {soac_{p}}
    {\texttt{transpose($soac_{c}$)}}
    {\texttt{transpose($soac_{r}$)}}
  } & \tagsc{Fuse-Transposed-Consumer}
\end{align*}

\begin{align*}
  \inference{
    \fusesto
    {os}
    {soac_{p}}
    {soac_{c}}
    {soac_{r}}
  }{
    \fusesto
    {os}
    {soac_{p}}
    {\texttt{reshape($shape$, $soac_{c}$)}}
    {\texttt{reshape($shape$, $soac_{r}$)}}
  } & \tagsc{Fuse-Reshaped-Consumer}
\end{align*}

The intuition behind these rules is that fusion is not sensitive to
what happens with the output of the consumer.

We will need a judgement to determine which SOAC inputs are simply
transformations of an origin array.  The judgement
\[
\boxed{
  \Re(e) = v
}
\]
means that $e$ is an application of \texttt{transpose} or
\texttt{reshape} (possibly both, or several in sequence) of the
array-typed variable $v$.  For example $e$, may be
\texttt{transpose($v$)}.  Valid judgements are defined by the
following inference rules.

\[
\inference{
  \Re(e) = v
}{
  \Re(\texttt{transpose}(k,n,e)) = v
}
\]

\[
\inference{
  \Re(e) = v
}{
  \Re(\texttt{reshape}(shape,e)) = v
}
\]

\[
\inference{
  \mbox{If $v$ is a variable}
}{
  \Re(v) = v
}
\]

When $\Re(e) = v$, we will say that the \textit{source array} of $e$
is $v$.

\section{Fusing across \texttt{transpose}}

In the general case, \texttt{transpose} acts like a fusion blocker -
if the output of a producer is transformed before being fed to the
consumer, then most likely fusion cannot take place.  In some
instances, we can perform a local transformation, either
\textit{pushing} the transposition past the consumer, or
\textit{pulling} it to before the producer.  An example is seen on
\cref{fig:fuse-across-transpose} - on \cref{fig:transpose-unfusible},
the transposition blocks fusion, but by pushing the transpose
operation to the return value of the consumer, as on
\cref{fig:transpose-fusible}, we expose
\texttt{map}-\texttt{map}-fusibility.  \Cref{fig:pull-transpose}
demonstrates same concept, but by pulling instead of pushing the
transposition.

\begin{figure}
\begin{subfigure}[t]{.5\textwidth}
\begin{colorcode}
let \{\emphh{b}\} = map(f, a) in
map\(\myindu{2}\)(g, \emp{transpose}(\emphh{b}))
\end{colorcode}
\subcaption{Unfusible\label{fig:transpose-unfusible}}
\end{subfigure}%
\begin{subfigure}[t]{.5\textwidth}
\begin{colorcode}
let \{\emphh{b}\} = map(f, a) in
\emp{transpose}(map\(\myindu{2}\)(g, \emphh{b}))
\end{colorcode}
\subcaption{Fusible\label{fig:transpose-fusible}}
\end{subfigure}

\caption{Pushing transposition past consumer}
\label{fig:fuse-across-transpose}
\end{figure}

\begin{figure}
\begin{subfigure}[t]{.5\textwidth}
\begin{colorcode}
let \emphh{b} = map\(\myindu{2}\)(f, a) in
map(g, \emp{transpose}(\emphh{b}))
\end{colorcode}
\subcaption{Unfusible\label{fig:pull-transpose-unfusible}}
\end{subfigure}%
\begin{subfigure}[t]{.5\textwidth}
\begin{colorcode}
let \emphh{b} = map\(\myindu{2}\)(f, \emp{transpose}(a)) in
map(g, \emphh{b})
\end{colorcode}
\subcaption{Fusible\label{fig:pull-transpose-fusible}}
\end{subfigure}

\caption{Pulling transposition before producer}
\label{fig:pull-transpose}
\end{figure}

To get an intution for the validity of these transformations, we
employ the concept of \textit{transpose depth}.  A standard textbook
transposition interchanges the outer two dimensions - hence we say
that it has a depth of $2$, as those are the dimensions that are
affected.  If a SOAC does not directly access the two outermost
dimensions, as for example a \texttt{mapT$\myindu{2}$} does not, we
can interchange them without modifying the values that are seen by the
body of the SOAC.  This can also be generalised to support the
generalised $(k,n)$-transposition described in \cref{sec:l0-reference}.

Specifically, \texttt{mapT$\myindu{n}$} accesses the element at index
\([i_{1}, \ldots, i_{n}]\), but we are free to transpose the indices
$i_{1}, \ldots, i_{n-1}$.  Of course, the order of the results will be
different, which is why we need to perform an inverse transposition on
the result.  To this end, \cref{fig:depth-of-transpose} defines a
function for determining the transpose depth of a $(k,n)$-transpose,
and \cref{fig:inverse-transpose} defines a notation for the inverse of
a transposition.

\begin{figure}
\begin{center}
  \[
  \mathcal{D}(k,n) =
  \begin{cases}
    k+n & n \geq 0\\
    k   & n < 0
  \end{cases}
  \]
\end{center}
\caption{Transposition depth}
\label{fig:depth-of-transpose}
\end{figure}
\begin{figure}
\begin{center}
\texttt{transpose\mymath{\myindu{-1}}(k,n,e)} $\equiv$ \texttt{transpose(k+n,-n,e)}
\end{center}
\caption{Inverse transpose}
\label{fig:inverse-transpose}
\end{figure}

Note the crucial property \texttt{transpose\(\myindu{-1}\)($k$, $n$,
  transpose($k$, $n$, $e$))~=~$e$}.

Our presentation will assume that transpositions are inlined as part
of the inputs to the consumer, as on \cref{fig:fuse-across-transpose}.

\subsection{Pushing \texttt{transpose}}

To get an intution for how the transformation works, let us look at a
slightly more complicated example:

\begin{colorcode}
let \{\emphh{b}\} = mapT(f, a) in
map\(\myindu{\mathcal{D}(k,n)}\)(g, \emp{transpose}(k, n, \emphh{b}), c)
\end{colorcode}

The consumer takes two inputs, and only one of them is from the
producer.  Our goal is to remove the \texttt{transpose} from $b$ - we
do not care about whether the algorithm will eventually try to fuse
$c$ as well.  Observe that by applying an inverse transposition to
both inputs, we would remove the transposition surrounding \texttt{b},
thus obtaining inputs \texttt{b} and
\texttt{transpose\(\myindu{-1}\)(k, n, c)}, and thereby exposing
\texttt{map}-\texttt{map}-fusibility between the producer and
consumer.  Of course, we still have to transpose the output of the new
consumer.  This solution also works \textit{only} when all inputs
coming from the producer are transposed in the \textit{exact same}
way, as otherwise applying the inverse transposition would not cause
the transpositions to go away.

To define fusion rules that fit into the framework established in
\cref{chap:fusion}, we will need a bit of machinery.  First, we define
a judgement for determining whether the outputs of a producer are
transposed by the consumer:

\[
\boxed{
  \text{transposed}(os,\overline{es}) \Rightarrow (k,n)
}
\]

Here, $os$ must be the outputs of a producer, and $\overline{es}$ are
the inputs of the consumer.  The judgement produces $(k,n)$ if one of
the outputs are transposed in $\overline{es}$, defined by the
following inference rules:

\[
\inference{
  \Re(e) = v
  &
  (v \in os)
}{
  \text{transposed}(os, \texttt{transpose($k$,$n$,$e$)}, \overline{es}) \Rightarrow (k,n)
}
\]

\[
\inference{
  \Re(e) = v
  &
  \text{transposed}(os,\overline{es}) \Rightarrow (k,n)
  &
  (v \notin os)
}{
  \text{transposed}(os, e, \overline{es}) \Rightarrow (k,n)
}
\]

We also need a judgement for checking whether all inputs coming from
the producer are transposed the exact same way, and if so, produce a
modified input list with inverse transpositions applied:

\[
\boxed{
  \text{transpose}^{-1}(k, n, os, \overline{es}) \Rightarrow \overline{es'}
}
\]

This judgement states that all inputs in $\overline{es}$ whose
underlying array is in $os$ is $(k,n)$-transposed.  Furthermore, the
result of applying an inverse $(k,n)$-transposition to every input in
$es$ is $\overline{es'}$.  The judgement is defined by the following
inference rules:

\[
\inference{
}{
  \text{transpose}^{-1}(k, n, os,\texttt{transpose($k$,$n$,$e$)}) \Rightarrow e
}
\]

\[
\inference{
  (\text{$e$ is not a transposition})
}{
  \text{transpose}^{-1}(k, n, os, e) \Rightarrow e
}
\]

\[
\inference{
  \Re(e) = v
  &
  \text{transpose}^{-1}(k, n, os, \overline{es}) \Rightarrow \overline{es'}
  &
  (v \in os)
}{
  \text{transpose}^{-1}(k, n, os, \texttt{transpose}(k, n, e), \overline{es}) \Rightarrow e, \overline{es'}
}
\]

\[
\inference{
  \Re(e) = v
  &
  \text{transpose}^{-1}(k, n, os, \overline{es}) \Rightarrow \overline{es'}
  &
  (v \in os)
}{
  \text{transpose}^{-1}(k, n, os, \texttt{transpose}(k,n,e), \overline{es}) \Rightarrow e, \overline{es'}
}
\]

\[
\inference{
  \Re(e) = v
  &
  \text{transpose}^{-1}(k, n, os, \overline{es}) \Rightarrow \overline{es'}
  &
  (v \notin os)
}{
  \text{transpose}^{-1}(k, n, os, e, \overline{es}) \Rightarrow \texttt{transpose$^{-1}$($k$, $n$, $e$)}, \overline{es'}
}
\]

We can now define a fusion rule for pushing \texttt{transpose} past
\texttt{mapT} operations of sufficient nesting.  The general outline
is as follows: First, determine whether an output of the producer is
$(k,n)$-transposed by the consumer.  Then, check whether \textit{all}
of the producer outputs used by the consumer are $(k,n)$-transposed,
and if so, apply the inverse transposition to every input, obtaining
$\overline{es'}$.  Finally, attempt to fuse the result.

\begin{align*}
\inference{
  \text{transposed}(os,\overline{es}) \Rightarrow (k,n)
  \\
  \text{transpose}^{-1}(k,n,os,es) \Rightarrow es'
  \\
  \fusesto
  {os}
  {soac_{p}}
  {\texttt{mapT$^{\mathcal{D}(k,n)}$($f$,$\overline{es'}$)}}
  {soac_{r}}
}{
  \fusesto
  {os}
  {soac_{p}}
  {\texttt{mapT$^{\mathcal{D}(k,n)}$($f$,$\overline{es}$)}}
  {\texttt{transpose($k$, $n$, $soac_{r}$)}}
} & \tagsc{Push-Transpose}
\end{align*}

\subsection{Pulling \texttt{transpose}}

Consider the following program:

\begin{colorcode}
let \{\emphh{b1}, \emphh{b2}\} = mapT\(\myindu{\mathcal{D}(k,n)}\)(f, a1, a2) in
reduceT(g, \emp{transpose}(k, n, \emphh{b1}), \emp{transpose}(k, n, \emphh{b2}), c)
\end{colorcode}

We can only push transpositions past sufficiently deeply nested
\texttt{mapT}s, and in this case the consumer is a \texttt{reduceT}.
However, since all of the inputs derived from outputs of the consumer
are $(k,n)$-transposed, we can instead transform the producer itself
by $(k,n)$-transposing its inputs.  This produces the following
program, where \texttt{mapT}-\texttt{reduceT} fusion is possible:

\begin{colorcode}
let \{\emphh{b1}, \emphh{b2}\} = mapT\(\myindu{\mathcal{D}(k,n)}\)(f, \emp{transpose}(k, n, a1), \emp{transpose}(k, n, a2)) in
reduceT(g, \emphh{b1}, \emphh{b2}, c)
\end{colorcode}

Again, this relies on the fact that the body of the producer is
invariant with respect to the outer $\mathcal{D}(k,n)$ dimensions.

Again, we will need an ancillary judgement.

\[
\boxed{
  \text{untranspose}(k, n, os, \overline{es}) \Rightarrow \overline{es'}
}
\]

This judgement checks that all inputs in $\overline{es}$ that use an
input from $os$ are $(k,n)$-transposed, and produces a new sequence of
inputs $\overline{es'}$ where such transposes are removed.  It is
defined by the following inference rules.

\[
\inference{
}{
  \text{untranspose}(k, n, os,\texttt{transpose($k$,$n$,$e$)}) \Rightarrow e
}
\]

\[
\inference{
  (\text{$e$ is not a transposition})
}{
  \text{untranspose}(k, n, os, e) \Rightarrow e
}
\]

\[
\inference{
  \Re(e) = v
  &
  \text{untranspose}(k, n, os, \overline{es}) \Rightarrow \overline{es'}
  &
  (v \in os)
}{
  \text{untranspose}(k, n, os, \texttt{transpose}(k, n, e), \overline{es}) \Rightarrow e, \overline{es'}
}
\]

\[
\inference{
  \Re(e) = v
  &
  \text{untranspose}(k, n, os, \overline{es}) \Rightarrow \overline{es'}
  &
  (v \notin os)
}{
  \text{untranspose}(k, n, os, e, \overline{es}) \Rightarrow e, \overline{es'}
}
\]

We can now define a fusion rule for pulling transpositions past
\texttt{mapT} producers of sufficient nesting.  The general outline is
as follows: First, determine whether an output of the producer is
$(k,n)$-transposed by the consumer.  Then, check whether \textit{all}
of the producer outputs used by the consumer are $(k,n)$-transposed,
and if so, strip those transpositions and $(k,n)$-transpose the inputs
to the producer instead.  Finally, attempt to fuse the result.

\begin{align*}
\inference{
  \text{transposed}(os,\overline{es_{c}}) \Rightarrow (k,n)
  &
  \text{untranspose}(k,n,os,\overline{es_{c}}) \Rightarrow \overline{es_{c}'}
  \\
  \fusesto
  {os}
  {\texttt{mapT$^{\mathcal{D}(k,n)}$($f$,transpose($k$,$n$,$\overline{es_{p}}$))}}
  {\texttt{mapT($g$, $\overline{es_{c}'}$)}}
  {soac_{r}}
}{
  \fusesto
  {os}
  {\texttt{mapT$^{\mathcal{D}(k,n)}$($f$,$\overline{es_{p}}$)}}
  {\texttt{mapT($g$, $\overline{es_{c}}$)}}
  {soac_{r}}
} & \tagsc{Pull-Transpose}
\end{align*}

\section{Fusing across \texttt{reshape}}

The idea behind fusing across \texttt{reshape} operations is similar
to the one covering \texttt{transpose}, although less well-developed.
We support solely \textit{pulling} reshape prior to \texttt{mapT}
producers taking single-dimensional arrays as input.  For example, we
can fuse the following program:

\begin{colorcode}
let \{\emphh{b}\} = mapT(f, a) in \emp{// a is one-dimensional}
reduceT(g, reshape((\(e\myindx{1}, \ldots, e\myindx{n}\)), \emphh{b}))
\end{colorcode}

Conceptually, the producer applies the function \texttt{f} to every
element in \texttt{a}.  We reshape \texttt{a} to be $n$-dimensional
and change the producer to be a depth-$n$ nest:

\begin{colorcode}
let \{\emp{b}\} = mapT\(\myindu{n}\)(f, reshape((\(e\myindx{1}, \ldots, e\myindx{n}\)) a)) in
reduceT(g, \emp{b})
\end{colorcode}

In the resulting program, \texttt{mapT}-\texttt{reduceT}-fusability is
exposed.

Again, we need ancillary judgements.  These are entirely analogous to
the ``\text{transposed}'' and ``\text{untransposed}'' judgements.

\[
\boxed{
  \text{reshaped}(os,\overline{es}) \Rightarrow (k,n)
}
\]

\[
\inference{
  \Re(e) = v
  &
  (v \in os)
}{
  \text{reshaped}(os,\texttt{reshape($shape$,$e$)}) \Rightarrow shape
}
\]

\[
\inference{
  \Re(e) = v
  &
  \text{reshaped}(os,\overline{es}) \Rightarrow shape
  &
  (v \notin os)
}{
  \text{reshaped}(os, e, \overline{es}) \Rightarrow shape
}
\]

\[
\boxed{
  \text{unreshape}(shape, os, \overline{es}) \Rightarrow \overline{es'}
}
\]

\[
\inference{
}{
  \text{unreshape}(shape, os,\texttt{reshape($shape$,$e$)}) \Rightarrow e
}
\]

\[
\inference{
  (\text{$e$ is not a reshaping})
}{
  \text{unreshape}(shape, os, e) \Rightarrow e
}
\]

\[
\inference{
  \Re(e) = v
  &
  \text{unreshape}(shape, os, \overline{es}) \Rightarrow \overline{es'}
  &
  (v \in os)
}{
  \text{unreshape}(k, n, os, \texttt{reshape}(shape, e), \overline{es}) \Rightarrow e, \overline{es'}
}
\]

\[
\inference{
  \Re(e) = v
  &
  \text{unreshape}(shape, os, \overline{es}) \Rightarrow \overline{es'}
  &
  (v \notin os)
}{
  \text{unreshape}(shape, os, e, \overline{es}) \Rightarrow e, \overline{es'}
}
\]

The fusion rule is also extremely similar to
\textsc{Pull-Transpose}.

\begin{align*}
\inference{
  \text{reshaped}(os, \overline{es_{c}}) \Rightarrow (e^{s}_{1}, \ldots, e^{s}_{n})
  &
  \text{unreshaped}((e^{s}_{1}, \ldots,e^{s}_{n}), os, \overline{es_{c}}) \Rightarrow \overline{es_{c}'}
  \\
  (\text{All of $\overline{es_{p}}$ have rank 1})
  &
  \fusesto
  {os}
  {\texttt{mapT$^{n}$($f$, reshape($(e^{s}_{1},\ldots,e^{s}_{n})$, $\overline{es_{p}}$))}}
  {\texttt{mapT($g$, $\overline{es_{c}'}$)}}
  {soac_{r}}
}{
  \fusesto
  {os}
  {\texttt{mapT($f$, $\overline{es_{p}}$)}}
  {\texttt{mapT($g$, $\overline{es_{c}}$)}}
  {soac_{r}}
} & \tagsc{Pull-Reshape}
\end{align*}

\section{ISWIM - Interchange Scan With Inner Maps}

The fusion algebra for \texttt{scanT} is quite poor -- in particular,
it can never be fused as a producer.  In some cases however, we can
rewrite \texttt{scanT} to expose producer-fusibility.  This section
presents a high-level transformation that may enable fusion of
\texttt{scanT}.  Specifically, when the body of a \texttt{scanT}
operation consists of a nested \texttt{mapT}, we can interchange the
two loops and transpose both input and output.  A simple example to
demonstrate the intuitive idea is illustrated on
\cref{fig:iswim-example}.  Using Haskell-like notation, a
\texttt{scan} operation on a matrix in which the binary associative
operator is \texttt{zipWith $\odot$} has the same semantics as
transposing the matrix, mapping each of the rows, i.e., former
columns, with \texttt{scan $\odot$} and transposing back the result.

In principle, this transformation interchanges the \texttt{scanT} with
the inner \texttt{mapT}, hence ISWIM, with the result that the
transformed code can be executed as a segmented scan~\cite{segScan},
i.e., exploiting both levels of parallelism.  With \texttt{scanT} on
the outside, we would have to choose between the parallel
\texttt{scanT} and the parallel \texttt{mapT}.  Furthermore, pushing
the least parallel construct, i.e., \texttt{scanT}, at the innermost
position might reveal a deeper \texttt{mapT}-nest, e.g., if the
original \texttt{scanT} was inside a \texttt{mapT} itself, thus
increasing the depth of parallelism.  Finally, if the created
\texttt{mapT} nest exhibits enough parallelism, then the
\texttt{scanT} can be executed sequentially rather than in parallel.
In this way, the ISWIM transformation is not solely about enabling
fusion, but is worthwhile on its own.

\begin{figure}
\begin{subfigure}[t]{.6\textwidth}
\begin{colorcode}
scanT(fn [real] ([real] x, [real] y) =>
        mapT(op +, x, y),
      \{0.0, \ldots, 0.0\}, a)
\end{colorcode}
\vspace{0.5cm}
\subcaption{Unfusible \label{fig:iswim-unfusible}}
\end{subfigure}%
\begin{subfigure}[t]{.4\textwidth}
\begin{colorcode}
transpose(
  mapT (fn [real] ([real] x) =>
          scanT(op +, 0.0, x),
                transpose(a)))
\end{colorcode}
\subcaption{Potentially fusible \label{fig:iswim-fusible}}
\end{subfigure}
\caption{Interchange Scan With Inner Maps}
\label{fig:iswim-example}
\end{figure}

Two fusion rules are defined.  One where ISWIM is applied to the
producer, and one where it is applied to the consumer.

\begin{align*}
\inference{
  \fusesto
  {os}
  {
    soac_{p}
  }
  {
    \texttt{transpose(mapT(scanT($f$), $\{e^{v}_{1}, \ldots, e^{v}_{k}\}, e^{a}_{1}, \ldots, e^{a}_{k}$))}
  }
  {
    soac_{r}
  }
}{
  \fusesto
  {os}
  {
    soac_{p}
  }
  {
    \texttt{scanT}(\texttt{mapT}(f), \{e^{v}_{1}, \ldots ,e^{v}_{k}\}, e^{a}_{1}, \ldots, e^{a}_{k})
  }
  {
    soac_{r}
  }
} \tagsc{Fuse-ISWIM-Consumer}
\end{align*}

\begin{align*}
\inference{
  \fusesto
  {os}
  {
    \texttt{transpose(mapT(scanT($f$), $\{e^{v}_{1}, \ldots, e^{v}_{k}\}, e^{a}_{1}, \ldots, e^{a}_{k}$))}
  }
  {
    soac_{c}
  }
  {
    soac_{r}
  }
}{
  \fusesto
  {os}
  {
    \texttt{scanT}(\texttt{mapT}(f), \{e^{v}_{1}, \ldots ,e^{v}_{k}\}, e^{a}_{1}, \ldots, e^{a}_{k})
  }
  {
    soac_{c}
  }
  {
    soac_{r}
  }
} \tagsc{Fuse-ISWIM-Producer}
\end{align*}

ISWIM depends critically on the fusion algorithm being able to fuse
through \texttt{transpose}.  There is also a further generalisation
for ISWIM, illustrated on \cref{fig:iswim-deep}, which permits the
inner \texttt{mapT} to be arbitrarily nested, but it has not yet been
implemented in the \LO{} compiler.

\begin{figure}[bt]
\begin{colorcode}
\emp{scanT}( fn ( [\cindx{1}[\cindx{...n+1}\mymath{\alpha\myindx{1}}]], ..., [\cindx{1}[\cindx{...n+1}\mymath{\alpha\myindx{k}}]] )
          ( [\cindx{1}[\cindx{...n+1}\mymath{\alpha\myindx{1}}]] x\mymath{\myindu{1}\myindx{1}}, ..., [\cindx{1}[\cindx{...n+1}\mymath{\alpha\myindx{k}}]] x\mymath{\myindu{1}\myindx{k}},
            [\cindx{1}[\cindx{...n+1}\mymath{\alpha\myindx{1}}]] x\mymath{\myindu{2}\myindx{1}}, ..., [\cindx{1}[\cindx{...n+1}\mymath{\alpha\myindx{k}}]] x\mymath{\myindu{2}\myindx{k}} ) =>
                \emphh{mapT}\cindu{n+1}(\mymath{\oplus}, x\mymath{\myindu{1}\myindx{1}},..., x\mymath{\myindu{1}\myindx{k}}, x\mymath{\myindu{2}\myindx{1}},..., x\mymath{\myindu{2}\myindx{k}}),
     (ne\cindx{1}, ..., ne\cindx{k}), a\cindx{1}, ..., a\cindx{k})
             \emphh{\(\equiv\)}
let (..., re\cindx{t}, ...) = (..., map\cindx{n+1}( replicate(1), ne\cindx{t} ), ...)
// Replicate dimension \(n+1\) of neutral elements so mapT sizes match
let ( y\cindx{1}, ..., y\cindx{k} ) =
  \emphh{mapT}(fn ( [\cindx{1}[\(\myindx{...n+1}\alpha\myindx{1}\)]],    ..., [\cindx{1}[\cindx{...n+1}\(\alpha\myindx{k}\)]] )
          ( [\cindx{1}[\cindx{...n+1}\(\alpha\myindx{1}\)]] x\cindx{1}, ..., [\cindx{1}[\cindx{...n+1}\(\alpha\myindx{k}\)]] x\cindx{k} ) =>
          \emphh{mapT}\cindu{n}(fn (\(\alpha\myindx{1}\),  ..., \(\alpha\myindx{k}\))
                   (\(\alpha\myindx{1}\) e\cindx{1},...,\(\alpha\myindx{k}\) e\cindx{k},
                    \(\alpha\myindx{1}\) x\cindx{1},...,\(\alpha\myindx{k}\) x\cindx{k})
                     =>\emp{scanT}(\(\oplus\), e\cindx{1}[0],...,e\cindx{k}[0], x\cindx{1},...,x\cindx{k}),
                 re\cindx{1}, ..., re\cindx{k}, x\cindx{1}, ..., x\cindx{k} ),
       transpose(1,n+1,a\cindx{1}), ..., transpose(1,n+1,a\cindx{k}))
in (transpose(n+1, q\cindx{1}-(n+1), y\cindx{1}), ..., transpose(n+1, q\cindx{k}-(n+1), y\cindx{k}))
// transpose back the result; q\mymath{\myindx{t}} is the dimension of \mymath{\alpha\myindx{t}}
\end{colorcode}
\caption{Arbitrary-depth generalisation of ISWIM}
\label{fig:iswim-deep}
\end{figure}

\section{Fusing A Transposed Producer}

While an input program will never contain a producer of the form
\texttt{transpose($k$,$n$,$soac$)}, the ISWIM transformation may
create them.  To fuse these, we first move the transpositions to the
inputs of the consumer.

We need yet another ancillary judgement:

\[
\boxed{
  \text{transpose}(k, n, os, \overline{es}) \Rightarrow \overline{es'}
}
\]

This judgement wraps every input in $\overline{es}$ whose source array
is in $os$ in an $(k,n)$-transposition, producing $\overline{es'}$.
It is defined by the following inference rules:

\[
\inference{
  \Re(e) = v
  &
  (v \in os)
}{
  \text{transpose}(k, n, os, e) \Rightarrow \texttt{transpose($k$,$n$,$e$)}
}
\]

\[
\inference{
  \Re(e) = v
  &
  (v \notin os)
}{
  \text{transpose}(k, n, os, e) \Rightarrow e
}
\]

\[
\inference{
  \Re(e) = v
  &
  (v \in os)
  &
  \text{transpose}(k, n, os, e,\overline{es}) \Rightarrow \overline{es'}
}{
  \text{transpose}(k, n, os, e,\overline{es}) \Rightarrow \texttt{transpose($k$,$n$,$e$)}, \overline{es'}
}
\]

\[
\inference{
  \Re(e) = v
  &
  (v \notin os)
  &
  \text{transpose}(k, n, os, e,\overline{es}) \Rightarrow \overline{es'}
}{
  \text{transpose}(k, n, os, e,\overline{es}) \Rightarrow e, \overline{es'}
}
\]

We can now define a fusion rule.  First, we extract the array inputs
of the consumer (not formalised), then transpose those whose source
arrays come from the producer, producing $\overline{es'}$.  Finally,
we attempt fusion where we have substituted the original array inputs
in the consumer with $\overline{es'}$.

\begin{align*}
\inference{
  \text{Inputs of $soac_{c}$ is $\overline{es}$}
  &
  \text{transpose}(k, n, os,\overline{es}) \Rightarrow \overline{es'}
  \\
  \fusesto
  {os}
  {soac_{p}}
  {\text{$soac_{c}$ with inputs $\overline{es'}$}}
  {soac_{r}}
}{
  \fusesto
  {os}
  {\texttt{transpose($k$, $n$, $soac_{p}$)}}
  {soac_{c}}
  {soac_{r}}
} \tagsc{Fuse-Transposed-Producer}
\end{align*}

%%% Local Variables:
%%% mode: latex
%%% TeX-master: "thesis.tex"
%%% End:
