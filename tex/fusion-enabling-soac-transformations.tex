\chapter{Fusion-enabling SOAC transformations}
\label{chap:fusion-enabling-soac-transformations}

The fusion rules in \cref{sec:fusion-rules} cover only simple cases,
where the output of the producer is used directly by the consumer,
without any intermediary steps.  This means that the following
program, where the output of the producer is first passed through
\texttt{transpose}, cannot be fused.
\begin{colorcode}
let \{\emphh{b}\} = mapT(f, a) in
mapT(fn [int] ([int] r) => mapT(g, r), \emp{transpose}(\emphh{b}))
\end{colorcode}
However, it is actually possible to fuse this case by first moving the
transposition to after the consumer instead:
\begin{colorcode}
let \{\emphh{b}\} = mapT(f, a) in
\emp{transpose}(mapT(fn [int] ([int] r) => mapT(g, r), \emphh{b}))
\end{colorcode}
After this transformation, the simple \texttt{map}-\texttt{map} fusion
rule applies.

In many cases, such rewriting of a producer-consumer pair is necessary
before the simple fusion rules can apply.  Indeed, one might consider
the \textsc{Fuse-Map-Redomap} rule a particularly simple example of
such a rewriting.  Fortunately, these rewriting schemes can be
incorporated into the existing fusion framework simply by considering
them as additional inference rules.  The above rewriting specifically
can be defined as follows (using the nested map notation from
\cref{fig:nested-maps}):

\begin{align*}
  \nfrac{
    \fusesto
    {e_{b}}
    {\texttt{mapT($l_{a}$,$e_a$)}}
    {\texttt{mapT$^{2}$($l_{b}$,$e_b$)}}
    {soac}
  }{
    \fusesto
    {e_{b}}
    {\texttt{mapT($l_{a}$,$e_a$)}}
    {\texttt{mapT$^{2}$($l_{b}$,transpose($e_b$))}}
    {\texttt{transpose($soac$)}}
  }
  \tagsc{Fuse-Map-Transpose-Map-Single}
\end{align*}

\begin{figure}
\begin{center}
\begin{colorcode}
mapT\(\myindu{1}\)(f, a\(\myindx{1}\), \ldots , a\(\myindx{k}\)) \emphh{\(\equiv\)}
mapT(g, a\(\myindx1\), \ldots, a\(\myindx{k}\))

mapT\(\myindu{n+1}\)(f, a\(\myindx{1}\), \ldots, a\(\myindx{k}\)) \emphh{\(\equiv\)}
mapT(fn \{[\(\beta\myindx{1}\)], \ldots, [\(\beta\myindx{t}\)]\} ([\(\alpha\myindx{1}\)] x\(\myindx{1}\), \ldots, [\(\alpha\myindx{k}\)] x\(\myindx{k}\))) =>
       mapT\(\myindu{n}\)(f, x\(\myindx{1}\), \ldots, x\(\myindx{k}\))
\end{colorcode}
\end{center}
\caption{Nested map notation}
\label{fig:nested-maps}
\end{figure}

Of course, this rule is far too specific - it covers only
\texttt{map}-producers and -consumers, and with a single output and
input respectively.  In the next section, we will see a more general
treatment of when we can fuse across arrays.

Furthermore, we will need to extend the SOAC notation we use for
expressing fusion judgements.  The
\textsc{Fuse-Map-Transpose-Map-Single} rule uses \texttt{transpose} in
places where we have previously considered only plain variables and
SOAC expression.  However, there is fundamentally nothing dubious
about this usage, and we will not need to modify the previously
defined rules.\fixme{Rephrase ``dubious''}

We will need a judgement to determine which SOAC inputs are simply
transformations of an origin array.  The judgement
\[
\boxed{
  \Re(e) = v
}
\]
means that $e$ is a transformation of the array-typed variable $v$.
For example $e$, may be \texttt{transpose($v$)}.  Valid judgements are
defined by the following inference rules.

\begin{align*}
\nfrac{
  \Re(e) = v
}{
  \Re(\texttt{transpose}(k,n,e)) = v
}
\end{align*}

\begin{align*}
\nfrac{
  \Re(e) = v
}{
  \Re(\texttt{reshape}(shape,e)) = v
}
\end{align*}

\begin{align*}
  \nfrac{
  }{
    \Re(v) = v
} \quad (\mbox{If $v$ is a variable})
\end{align*}

\section{ISWIM - Interchange scan with inner maps}

The fusion algebra for \texttt{scanT} is quite poor -- in particular,
it can never be fused as a producer.  In some cases, however, we can
rewrite \texttt{scanT} to expose producer-fusability.  Specifically,
when the body of a \texttt{scanT} operation consists of a nested
\texttt{mapT}, we can interchange the two loops and transpose both
input and output.  A simple example is as follows:\fixme{Finish}

\begin{colorcode}
scanT(fn [real] ([real] x, [real] y) =>
        mapT(op +, x, y),
     , \{0.0, \ldots, 0.0\}, a)   \emphh{\mymath{\equiv}}
transpose(mapT (fn [real] ([real] x) =>
                  scanT(op +,0.0,x)
               , transpose(a) )
\end{colorcode}

\section{Fusing across \texttt{reshape}}

\section{Fusing across \texttt{transpose}}

Push-transpose example:

\begin{colorcode}
let \{\emphh{b}\} = mapT(f, a) in
mapT\(\myindu{2}\)(g, \emp{transpose}(\emphh{b}))
  \emphh{\(\Downarrow\)}
let \{\emphh{b}\} = mapT(f, a) in
\emp{transpose}(mapT\(\myindu{2}\)(g, \emphh{b})
\end{colorcode}

\[
\boxed{
  \text{untranspose}(k,n,os,es) \Rightarrow es'
}
\]

\begin{align*}
\nfrac{
}{
  \text{untranspose}(k,n,os,()) \Rightarrow ()
}
\end{align*}

\begin{align*}
& \nfrac{
  \Re(e) = v
  \quad
  \text{untranspose}(k,n,os,es) \Rightarrow es'
}{
  \text{untranspose}(k,n,os,(\texttt{transpose}(k,n,e),es)) \Rightarrow (e,es')
} & \quad (v \in os)
\\
\\
& \nfrac{
  \Re(e) = v
  \quad
  \text{untranspose}(k,n,os,es) \Rightarrow es'
}{
  \text{untranspose}(k,n,os,(e,es)) \Rightarrow (e,es')
} & \quad (v \notin os)
\end{align*}

And rule:

\begin{align*}
  \nfrac{
    \fusesto
    {os}
    {soac_{p}}
    {\texttt{mapT$^{k+n}$($f$,$\overline{es'}$)}}
    {soac_{r}}
    \quad
    \text{untranspose}(k,n,os,es) \Rightarrow es'
  }{
    \fusesto
    {os}
    {soac_{p}}
    {\texttt{mapT$^{k+n}$($f$,transpose(k, n, $es$))}}
    {\texttt{transpose$^{-1}$(k, n, $soac_{r}$)}}
  }
  \tagsc{Push-Transpose}
\end{align*}

Pull-transpose example:

\begin{colorcode}
let \{\emphh{b}\} = mapT\(\myindu{2}\)(f, a) in
reduceT(g, v, \emp{transpose}(\emphh{b}))
  \emphh{\(\Downarrow\)}
let \{\emphh{b}\} = mapT\(\myindu{2}\)(f, \emp{transpose}(a)) in
reduceT(g, v, \emphh{b})
\end{colorcode}

And rule:

\[
\boxed{
  \text{transpose}(k,n,os,es) \Rightarrow es'
}
\]

\begin{align*}
\nfrac{
}{
  \text{transpose}(k,n,os,()) \Rightarrow ()
}
\end{align*}

\begin{align*}
& \nfrac{
  \Re(e) = v
  \quad
  \text{transpose}(k,n,os,es) \Rightarrow es'
}{
  \text{transpose}(k,n,os,(e,es)) \Rightarrow (\texttt{transpose}(k,n,e),es')
} & \quad (v \in os)
\\
\\
& \nfrac{
  \Re(e) = v
  \quad
  \text{transpose}(k,n,os,es) \Rightarrow es'
}{
  \text{transpose}(k,n,os,(e,es)) \Rightarrow (e,es')
} & \quad (v \notin os)
\end{align*}

\begin{align*}
  \nfrac{
    \text{transpose}(k,n,os,es_{p}) \Rightarrow es_{p}'
    \quad
    \fusesto
    {os}
    {\texttt{mapT$^{k+n}$($f$,$\overline{es_{p}'}$)}}
    {\texttt{mapT($g$, $\overline{es_{c}}$)}}
    {soac_{r}}
  }{
    \fusesto
    {os}
    {\texttt{mapT$^{k+n}$($f$,$\overline{es_{p}}$)}}
    {\texttt{mapT($g$, transpose(k, n, $\overline{es_{c}}$))}}
    {soac_{r}}
  }
  \tagsc{Pull-Transpose}
\end{align*}

\begin{figure}
\begin{center}
\texttt{transpose\mymath{\myindu{-1}}(k,n,e)} $\equiv$ \texttt{transpose(k+n,-n,e)}
\end{center}
\caption{Inverse transpose}
\label{fig:inverse-transpose}
\end{figure}

%%% Local Variables: 
%%% mode: latex
%%% TeX-master: "thesis.tex"
%%% End: 
