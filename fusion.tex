\chapter{Fusion}
\label{chap:fusion}

This chapter will outline the principles behind
\textit{producer-consumer} loop fusion, describe their implementation
in the \LO{} compiler, and discuss possible complications and
restrictions of my handling of loop fusion.

In producer-consumer fusion, the aim is to merge (or \textit{fuse})
two loops, where the output of the first loop -- the producer -- is
used as input to the second -- the consumer.  We currently only fuse
loops that are expressed via SOACs, not the \texttt{do}-notation.  The
reason for this is to simplify analysis, as it can be hard to
determine in which cases arbitrary \texttt{do}-loops can be combined,
whereas it is possible to define simple rules for how and when SOACs
can be fused.

As a simple case, we can fuse the two loops in
\[
\texttt{(map~$f$)~$\circ$~(map~$g$)}
\]
and get
\[
\texttt{map~$(f~\circ~g)$},
\]
thus removing the need to construct an intermediary array for the
result of \texttt{map~$g$}, and in the context of GPGPU, reducely the
likelyhood of global memory accesses.  We will write ``$c_{1}$-$c_{2}$
fusion'' for the case where a fusion is formed with $c_{1}$ as the
producer and $c_{2}$ as the consumer.  Therefore, the previous example
would be ``\texttt{map}-\texttt{map}''-fusion.

\section{Fusion in \LO{}}

The language used to describe the fusion algorithm in this chapter is
the internal \LO{} described in chapter \ref{chap:internal}, and we
will consider \texttt{replicate} to be a SOAC.\footnote{In fact,
  \texttt{replicate(n,x)} can be written as \texttt{map(fn t (int i)
    => x, iota(n))}, and futher investigation may reveal that keeping
  \texttt{replicate} as a separate construct has no benefit.}

\begin{figure}
  \begin{center}
    \begin{tabular}{c|c}
      \textbf{Producers} & \textbf{Consumers} \\\hline
      \texttt{map} & \texttt{map} \\\hline
      & \texttt{reduce} \\\hline
      \texttt{scan} & \texttt{scan} \\\hline
      \texttt{filter} & \texttt{filter} \\\hline
      & \texttt{redomap} \\\hline
      \texttt{replicate} & \\\hline
    \end{tabular}
  \end{center}
  \caption{Producers and consumers in \LO}
  \label{fig:producers-consumers}
\end{figure}

\fref{fig:producers-consumers} lists which \LO{} SOACs are producers,
which are consumers, and which are both.  In particular, note that
even if we have a \texttt{reduce}-expression returning an array, this
does not mean that the \texttt{reduce}-expression is a producer - it
can under no circumstances be fused into another SOAC-expression.  The
reason is that the output of the reduction is only fully known after
the final input array element has been processed.  Consider the
following program:

\begin{colorcode}
let b = reduce(fn [int] ([int] acc, int x) =>
                 map(op + (x), acc),
               iota(10), a) in
map(f, b)
\end{colorcode}

The contents of the array \texttt{b} is not determined until the very
last element of \texttt{a} has been processed, and thus fusion with
the \texttt{map}-expression cannot take place.  While it is possible
to use \texttt{reduce} in a way that could theoretically be fused with
a consumer (for example by using it to simulate \texttt{map}), the
analysis necessary to determine whether a given reduction is fusable
would be quite onerous, and likely not useful in any but contrived
examples, such as the above-mentioned simulation of \texttt{map}.

In this way, \texttt{reduce} differs from \texttt{map}, in which each
element of the output is calculated from one element of the input ---
a classic case of data-parallelism.

Even if we have a producer-consumer-pair, not all such pairs
\textit{can} be fused, and not all are \textit{desirable} to fuse.
For instance, \texttt{filter}-\texttt{map} fusion is not possible,
although \texttt{filter}-\texttt{reduce} is.  The reason is that the
size of the \texttt{map}-output is the same as the size of its input,
yet the size of the output of \texttt{filter} cannot be known in
advance, which preclude an efficient fused form.

The \textit{fusion algebra} in section \ref{sec:fusionalgebra}
describes in detail which producer-consumer pairs can be fused, as
well as the form of the resulting SOAC.

Section \ref{sec:invalidfusion} goes into greater detail on invalid
fusion, and section \ref{sec:whentofuse} covers cases in which fusion
may actually be detrimental to performance.

...

My structural analysis is rooted in the
T$_{1}$-T$_{2}$-transformation~\cite{red_dragon}.\fxnote{Insert more
  about the transformation.}

\section{Invalid fusion}
\label{sec:invalidfusion}

Mumble mumble, something about uniqueness types.\fxnote{Finish this.}

\section{When to fuse}
\label{sec:whentofuse}

Even when fusion is possible, it may not be beneficial, and may be
harmful to overall performance in the following cases.

\begin{description}[style=nextline]
\item[Computation may be duplicated.]

In the program
\begin{colorcode}
let x = map(f, a) in
\{map(g, x), map(h, x)\}
\end{colorcode}
fusing the \texttt{x}-producer into the two consumers will double the
number of calls to the function \texttt{f}, which might be expensive.
The implementation in the \LO{} compiler will currently only fuse if
absolutely no computation is duplicated, although this is likely too
conservative.  Duplicating cheap work, for example functions that use
only primitive operations on scalars, is probably not harmful to
overall performance, although I have not investigated this
fully.\fxnote{Reference someone who has.}

In general, in the context of GPU, the tradeoff between duplicating
computation and increasing communication is not an easy problem to
solve.  Accessing global memory can be more than a hundred times
slower than accessing local (register) memory, hence duplicating
computation is in many cases preferable.

\item[Can reduce memory locality.]

  Consider a simple case of fusing
  \texttt{(map~$f$)~$\circ$~(map~$g$)}.  When $g$ is executed for an
  element of the input array, neighboring elements will be put into
  the cache, making them faster to access.  This exhibits good data
  locality.  In contrast, the composed function $f~\circ~g$ will
  perform more work after accessing a given input element, increasing
  the risk that the input array may be evicted from the cache before
  the next element is to be used.  On GPUs, there is the added risk of
  the kernel function exercising additional register pressure, which
  may reduce hardware occupancy (thus reducing latency hiding) by
  having fewer computational cores active.  In this case, it may be
  better to execute each of the two \texttt{map}s as separate kernels.

  The \LO{} compiler does not currently handle this problem, as it is
  envisioned that a later (and as-of-yet unimplemented) transformation
  will perform \textit{loop distribution} (sometimes called
  \textit{loop fission}).  This step is necessary anyway, as it can be
  used to improve the degree of parallelism, compared to the original
  program.\fixme{Insert example.}
\end{description}

\section{Dataflow}

\newcommand{\unfusable}[0]{\textsc{unfusable}}
\newcommand{\inputs}[0]{\textsc{arrInputs}}
\newcommand{\soacs}[0]{\textsc{SOACs}}
\newcommand{\patNames}[1]{\textsc{patNames}(#1)}
\newcommand{\childExps}[1]{\textsc{childExps}(#1)}
\newcommand{\parentExp}[1]{\textsc{parentExp}(#1)}

We will track the following pieces of information.\fxnote{All of this can be done in one pass - explain how and why.}

\begin{itemize}
\item The set of all SOACs in the program, modelled as a mapping from
  a (unique) identifier to a pair of a SOAC expression and its output
  pattern.  We shall say $\soacs(e)$ to refer to this mapping, and
  $\soacs(e)[l]$ to refer to the SOAC with label $l$.  For example,
  \begin{align*}
  & \soacs(\texttt{let \{$a$,$b$,$c$\} = map2($f$,$x$,$y$,$z$) in \{$a$,$b$,$c$\}}) =\\
  & \quad \{ (\ell, (\texttt{\{$a$,$b$,$c$\}},\texttt{map2($f$,$x$,$y$,$z$)})) \},
  \end{align*}
  where $\ell$ is some unique label.

\item The \textit{unfusable set}, which is key to preventing unwanted
  fusion, as it indicates which SOACs should never be fused.  The
  unfusable set prevents both undesired and invalid fusion, as
  outlined in sections \ref{sec:whentofuse} and
  \ref{sec:invalidfusion} respectively.  Given an \LO{} expression
  $e$, we shall say $\unfusable(e)$ to refer to the unfusable set
  produced by $e$.  For example:
  \begin{align*}
  \unfusable(&\texttt{let x = map2(f,a) in}\\
  &\texttt{let y = map2(g,a) in \{x,y\}}) = \{a\},
  \end{align*}
  because $a$ is used twice, and hence fusing its producer into
  \texttt{f} and \texttt{g} would cause work duplication.  (To
  simplify the example, I have ignored $f$ and $g$ when computing the
  unfusable set, although as we shall see below, this is not the case
  in practice.)

\item A mapping from arrays to a set of the SOACs that use the array
  as input.  This is modelled as a set of pairs, each pair consisting
  of an array name and a SOAC name.  We shall refer to the mapping
  generated by a given expression $e$ as $\inputs(e)$.  For example,
  \[
  \inputs(\texttt{map2($f$, $x$, $y$, $z$)}) = \{ (x, \{\ell\}), (y, \{\ell\}), (z, \{\ell\}) \},
  \]
  where $\ell$ is the label of the \texttt{map2}-SOAC.

  We define an associative and commutative operation $\sqcup$ to
  combine input mappings as follows.
  \begin{align*}
  &\{(v_{1},s_{1}),\ldots,(v_{n},s_{n})\} \sqcup \{(v_{n+1},s_{n+1}),\ldots,(v_{n+m},s_{n+m})\} =\\
  &\quad \{(v_{i}, \bigcup_{(v_{i},s)} s)\},
  \end{align*}

  Intuitively, $x \sqcup y$ is a mapping that contains the union of
  the keys in $x$ and $y$, with the value for a key being the union of
  the values for that key in $x$ and $y$ (or just an untouched value,
  if the key was only present in one of the mappings).
\end{itemize}

If more specific rules are not given, the data flows default to the
following.

\begin{align*}
  \unfusable(e) &= \bigcup_{e'\in\childExps{e}} \unfusable(e') \\
  \\
  \inputs(e) &= \bigsqcup_{e'\in\childExps{e}} e'\\
  \\
  \soacs(e) &= \bigcup_{e'\in\childExps{e}} \soacs(e')
\end{align*}

Now for specific rules, based on the shape of the given expression.

\begin{description}[style=nextline]
\item[Case $e \equiv v$]

  This rule is only applied when $v$ is not an array input to a SOAC,
  which implies that the producer of $v$ cannot be fused.
\begin{align*}
  & \unfusable(e) = \{v\} \\
\end{align*}

\item[Case $e \equiv \texttt{$v$[$e_{1}$, \ldots, $e_{n}$]}$]

  \begin{align*}
  & \unfusable(e) = \{v\} \cup \bigcup_{1\leq i \leq n}\unfusable(e_{i}) \\
\end{align*}

\item[Case $e \equiv \texttt{if $e_{c}$ then $e_{t}$ else $e_{f}$}$]

  The \unfusable{} of a conditional consists of whatever is in the
  \unfusable{} sets of its branches, plus any SOAC outputs that may be
  used multiple times.  Note that an output can be used in both the
  true and the false branch, and it will still only have been
  considered to be used once.

\begin{align*}
  & \inputs(e) =\\
  & \quad \{ (v,s')\ |\ (v,s) \in \inputs(e_{t}) \cup \inputs(e_{f}) \}\\
  & \quad\quad \text{Where $s'$ is the set of all consumers of $v$ in both $e_{t}$ and $e_{f}$.} \\
  \\
  & \unfusable(e) =\\
  & \quad \unfusable(e_{c}) \cup \unfusable(e_{t}) \cup \unfusable(e_{f})\\
  & \quad \cup (\inputs(e_{c}) \cap \inputs(e_{t}))\\
  & \quad \cup (\inputs(e_{c}) \cap \inputs(e_{f}))\\
\end{align*}

\item[Case $e \equiv \texttt{let $vs$ = map2(fn $t$ ($t_{1}$ $v_{p_{1}}$, \ldots, $t_{n}$ $v_{p_{n}}$) => $e_f$, $e_{1}$, \ldots, $e_{n}$) in $e_{b}$}$]

  The big question here is whether the SOAC can be fused as producer
  with something in $\soacs(e_{b})$.  In the following, $\ell_{p}$ is
  a fresh name.

  \fxnote{Finish description.  Not sure how to explain this in a simple
    manner, there are a ton of rules, after all.  There's also the
    annoying issue of turning the SOAC array arguments into names...}

\begin{description}
\item[Cannot fuse:]
\begin{align*}
  & \unfusable(e) =\\
  & \quad \unfusable(e_{b})\\
  & \quad \cup \{ v\ |\ (v,s) \in \inputs(e_{f}) \} \\
  & \quad \cup \{ v\ |\ \textrm{$v$ is used as input here but is also in $\inputs(e_{b})$} \}\\
  \\
  & \inputs(e) =\\
  & \quad \inputs(e_{b}) \sqcup \inputs(e_{f})\\
  & \quad \sqcup \{(e_{1},\{\ell_{p}\}), \ldots, (e_{n},\{\ell_{p}\})\}\\
  \\
  & \soacs(e) = \soacs(e_{b}) \cup \{(\ell_{p}, (vs, \texttt{map2(\ldots)}))\} \\
  &
\end{align*}

\item[Can fuse with some SOAC $\ell_{c}$:]\hfill\\

  Let $(vs',s') = \textsc{doThatFunkyFusionThing}(\ell, \ell_{c})$.
  \fxnote{Magic happens -- I probably need to factor the precise fusion
    mechanics into a separate section, putting them inline here would
    be really confusing.}

\begin{align*}
  & \soacs(e) = (\soacs(e_{b}) \backslash \ell_{c}) \cup \soacs(e_{f}) \cup \{ (\ell_{c}, (vs', s')) \}\\
  \\
  & \inputs(e) =\\
  & \quad (\textrm{$\inputs(e_{b})$ with all mappings to $\ell_{c}$ removed}) \\
  & \quad \sqcup \{ (v,\ell_{c}) |\ \textrm{for all array inputs $v$ to $s'$} \}
\end{align*}
\end{description}

\item[Case $e \equiv \texttt{let $v_{1}$ = $v_{2}$ with [$e_{1}$,\ldots,$e_{n}$] <- $e_{v}$ in $e_b$}$]

Record the in-place modification of $v_{2}$. \fxnote{Not sure how to model this yet.}

\item[Case $e \equiv \texttt{loop ($p$ = $e_{1}$) = for $v$ < $e_{2}$ do $e_{3}$ in $e_{4}$}$]

  For loops, we add any arrays used as SOAC inputs in the loop body to
  the unfusable set, as fusing into the loop would duplicate
  computation.  This is similar to how we ban fusing into
  SOAC-functions.
\begin{align*}
  & \unfusable(e) =\\
  & \quad \unfusable(e_{1}) \cup \unfusable(e_{2}) \cup \unfusable(e_{3}) \cup \unfusable(e_{4}) \\
  & \quad \cup \{ v\ |\ (v,s) \in \inputs(e_{3}) \}
\end{align*}

\end{description}

\section{Fusion algebra}
\label{sec:fusionalgebra}

\fxnote{This is horrible.  Fix me.}

\begin{figure}[bt]
\begin{colorcode}
// \emp{replicate can be fused}
// \emp{without restrictions}
let x = replicate(N,a)in 
let y = map2(f, x, b) in
let z = map2(g, x, c) in 
let x[i] = ...
    \emphh{\mymath{\equiv}}
let x = replicate(N, a) in 
let y = map2( fn \mymath{\beta\myindx{1}} (\mymath{\alpha\myindx{1}} b\mymath{\myindx{i}}) 
              => f(a,b\mymath{\myindx{i}}), b)
let z = map2( fn \mymath{\beta\myindx{2}} (\mymath{\alpha\myindx{2}} c\mymath{\myindx{i}}) 
              => g(a,c\mymath{\myindx{i}}), c)
in let x[i] = ...   


//\emp{map2 o map2 \mymath{\Rightarrow} map2}  
let (x1, x2) = map2(f, a1)
in  map2(g, x1, y)   
    \emphh{\mymath{\equiv}}
map2(fn \mymath{\beta} (\mymath{\alpha\myindx{1}} a1\mymath{\myindx{i}}, \mymath{\alpha\myindx{2}} y\mymath{\myindx{i}})
  =>let (x1\mymath{\myindx{i}}, x2\mymath{\myindx{i}}) = f(a1\mymath{\myindx{i}})
    in  g(x1\mymath{\myindx{i}}, y\mymath{\myindx{i}})
, a1, y )


//\emp{reduce2 o map2\mymath{\Rightarrow}redomap2}
let (x1, x2) = map2(f, a1)
in  reduce2(\mymath{\oplus},e\mymath{\myindx{1}},e\mymath{\myindx{2}}, x1,y)   
    \emphh{\mymath{\equiv}}
redomap2(\mymath{\oplus}
, fn (\mymath{\beta\myindx{1}},\mymath{\beta\myindx{2}}) ( \mymath{\beta\myindx{1}} e\mymath{\myindx{1}}, \mymath{\beta\myindx{2}} e\mymath{\myindx{2}}
             , \mymath{\alpha\myindx{1}} a1\mymath{\myindx{i}},\mymath{\alpha\myindx{2}} y\mymath{\myindx{i}})
   => let (x1\mymath{\myindx{i}}, x2\mymath{\myindx{i}}) = f(a1\mymath{\myindx{i}})
      in  \mymath{\oplus}(e\mymath{\myindx{1}},e\mymath{\myindx{2}},x1\mymath{\myindx{i}},y\mymath{\myindx{i}})
, (e\mymath{\myindx{1}}, e\mymath{\myindx{2}}), a1, y )


//\emp{redomap2 o map2\mymath{\Rightarrow}redomap2}
let (x1, x2) = map2(f, a1)
in  redomap2(\mymath{\oplus}, g, e, x1, y)
    \emphh{\mymath{\equiv}}
redomap2(\mymath{\oplus}
, fn \mymath{\beta} (\mymath{\beta} e, \mymath{\alpha\myindx{1}} a1\mymath{\myindx{i}}, \mymath{\alpha\myindx{2}} y\mymath{\myindx{i}})
   => let (x1\mymath{\myindx{i}}, x2\mymath{\myindx{i}}) = f(a1\mymath{\myindx{i}})
      in  g(e, x1\mymath{\myindx{i}}, y\mymath{\myindx{i}})
, e, a1, y )

//\emp{filter2 o filter2\mymath{\Rightarrow}filter2}
//\emp{{\em{}IFF} consumer's input set}
//\emp{  \mymath{\subseteq} producer's output set}
let (x1,x2)=filter2(c\mymath{\myindx{1}},a1,a2)
in  let y = filter2(c\mymath{\myindx{2}}, x1) ..
    \emphh{\mymath{\equiv}}
let (y, dead) = filter2(
  fn bool (\mymath{\alpha\myindx{1}} a1\mymath{\myindx{i}},\mymath{\alpha\myindx{2}} a2\mymath{\myindx{i}})=> 
      if   c\mymath{\myindx{1}}(a1\mymath{\myindx{i}}, a2\mymath{\myindx{i}}) 
      then c\mymath{\myindx{2}}(a1\mymath{\myindx{i}}) 
      else false 
, a1, a2 ) ..

//\emp{reduce2 o filter2\mymath{\Rightarrow}redomap2}
//\emp{{\em{}IFF} consumer's input list}
//\emp{  \mymath{\equiv} producer's output list}
let x = filter2(c, a)
in  reduce2(\mymath{\oplus}, e, x)
    \emphh{\mymath{\equiv}}
reduce2(fn \mymath{\beta} (\mymath{\beta} e, \mymath{\beta} a\mymath{\myindx{i}}) =>
  if c(a\mymath{\myindx{i}}) then \mymath{\oplus}(e,a\mymath{\myindx{i}}) else e
, e, a )

//\emp{reduce2 o filter2\mymath{\Rightarrow}redomap2}
//\emp{{\em{}IFF} consumer's input set}
//\emp{  \mymath{\subseteq} producer's output set}
let (x1,x2)=filter2(c, a1, a2)
in  reduce2(\mymath{\oplus}, e, x1)
    \emphh{\mymath{\equiv}}
redomap2(\mymath{\oplus}
, fn \mymath{\beta} (\mymath{\beta} e, \mymath{\alpha\myindx{1}} a1\mymath{\myindx{i}}, \mymath{\alpha\myindx{2}} a2\mymath{\myindx{i}})
   => if c(a1\mymath{\myindx{i}}, a2\mymath{\myindx{i}})
      then \mymath{\oplus}(e, a1\mymath{\myindx{i}}) else e
, e, a1, a2 )

//\emp{redomap2 o filter2\mymath{\Rightarrow}redomap2}
//\emp{{\em{}IFF} consumer's input set}
//\emp{  \mymath{\subseteq} producer's output set}
let (x1,x2)=filter2(c, a1, a2)
in  redomap2(\mymath{\oplus}, g, e, x1)
    \emphh{\mymath{\equiv}}
redomap2(\mymath{\oplus}
, fn \mymath{\beta} (\mymath{\beta} e, \mymath{\alpha\myindx{1}} a1\mymath{\myindx{i}}, \mymath{\alpha\myindx{2}} a2\mymath{\myindx{i}})
   => if c(a1\mymath{\myindx{i}}, a2\mymath{\myindx{i}})
      then g(e, a1\mymath{\myindx{i}}) else e
, e, a1, a2 )
\end{colorcode}
\caption{Compositional Algebra For Fusion}
\label{fig:fusion-algebra}
\end{figure}

%%% Local Variables: 
%%% mode: latex
%%% TeX-master: "thesis.tex"
%%% End: 
