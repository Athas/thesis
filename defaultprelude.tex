\usepackage[english]{babel}
\usepackage[utf8]{inputenc}
\usepackage[T1]{fontenc}
\usepackage[british]{isodate}

% Layout Fixes
\usepackage{booktabs} % nicer spacing between table rulers
\usepackage{microtype}
\usepackage{fixltx2e} % To prevent the figures from being placed
                      % out-of-order with respect to their
                      % "non-starred" counterparts


\usepackage{amsmath,amssymb, amsbsy}
\usepackage[amsmath,amsthm,thmmarks]{ntheorem}
\usepackage{semantic, stmaryrd}
\usepackage{graphicx}
\usepackage[disable]{todonotes}

\presetkeys{todonotes}{inline}{}
\usepackage{adjustbox}
\usepackage{tikz}
\usetikzlibrary{trees}
\usepackage{algpseudocode}
\usepackage{algorithm}
\usepackage{float}
\newfloat{algorithm}{t}{lop}
\usepackage{listings}
\usepackage{pdfpages}
\usepackage{color}
\usepackage{fancyvrb}

\usepackage{enumitem} % Allows \begin{description}[style=nextline]

\usepackage{fixme}
\fxsetup{
    status=draft,
    author=,
    layout=margin,
    theme=color
}

\lstset{
  basicstyle=\small\ttfamily,
  breaklines=true, 
  language=Haskell, 
  keywords={map,parmap,foldl,zipWith,replicate,let,in,if,forall,then,else}, 
  escapechar=@,
  xleftmargin=.25in
% ,  identifierstyle=\idstyle
}

\makeatletter
\newcommand*\idstyle{%
        \expandafter\id@style\the\lst@token\relax
}
\def\id@style#1#2\relax{%
        \ifcat#1\relax\else
                \ifnum`#1=\uccode`#1%
                        \color{blue}
                \fi
        \fi
}
\makeatother

% ADD "parfor" to algorithmic environment
  % declaration of the new block
  \algblock{ParFor}{EndParFor}
  % customising the new block
  \algnewcommand\algorithmicparfor{\textbf{parfor}}
  \algnewcommand\algorithmicpardo{\textbf{do}}
  \algnewcommand\algorithmicendparfor{\textbf{end\ parfor}}
  \algrenewtext{ParFor}[1]{\algorithmicparfor\ #1\ \algorithmicpardo}
  \algrenewtext{EndParFor}{\algorithmicendparfor}

% Bibliography
%\usepackage[style=alphabetic,natbib=true]{biblatex}
\usepackage[hyperref,style=numeric,natbib=true, defernumbers=true]{biblatex}
\usepackage[hyperindex,hidelinks]{hyperref}

% Fonts
\usepackage{palatino}
\linespread{1.05}
\usepackage[scaled]{beramono}

% customize chapter pages
\makepagestyle{myheadings}
\makepagestyle{myheadingschapterpage}
\makeevenfoot{myheadingschapterpage}{}{\thepage}{}
\makeoddfoot{myheadingschapterpage}{}{\thepage}{}
\aliaspagestyle{chapter}{myheadingschapterpage}
\aliaspagestyle{title}{myheadingschapterpage}
\makeevenhead{myheadings}{\hskip.5cm\leftmark}{}{}
\makeoddhead{myheadings}{}{}{\rightmark\hskip.5cm}
\makeevenfoot{myheadings}{}{\thepage}{}
\makeoddfoot{myheadings}{}{\thepage}{}
\pagestyle{myheadings}

\def\thefigure{\arabic{figure}}
\setcounter{tocdepth}{0}


\setcounter{secnumdepth}{1}
\setcounter{chapter}{0}
\setsecheadstyle{\large\bfseries\raggedright}
\setsubsecheadstyle{\bfseries}

% BIBLIOGRAPHY styling
% Print the DOI url out
%\newcommand*{\doi}[1]{DOI: \url{http://dx.doi.org/#1}}

\DeclareFieldFormat{doi}{\textsc{doi}: \url{http://dx.doi.org/#1}}
\DeclareNameAlias{default}{last-first/first-last}
%\DeclareFieldFormat{title}{\emph{#1}}
\renewcommand\mkbibnamefirst[1]{\textsc{#1}}
\renewcommand\mkbibnamelast[1]{\textsc{#1}}
\renewcommand*{\bibfont}{\footnotesize}

% No page numbers for parts in TOC
\cftpagenumbersoff{part}
% Footnote symbols
%\renewcommand*{\thefootnote}{\fnsymbol{footnote}}

% Figures
\newsubfloat{figure}

\usepackage{sidecap}
\usepackage{caption}
\captionsetup{margin=0pt, font=small, labelfont=bf, format=hang}
\setlength{\abovecaptionskip}{0pt}
\setlength{\belowcaptionskip}{0pt}


% Default commands
\newcommand{\subimgwidth}{.48\textwidth}
\newcommand{\imgwidth}{.85\textwidth}


\newtheorem{example}{Example}
%
\theoremstyle{plain}
\theoremsymbol{\tiny $\Box$}
\newtheorem{definition}[equation]{Definition}


% Bitwise and and or operators
% http://tex.stackexchange.com/questions/39313/double-nested-logical-and-and-logical-or-symbols
\DeclareFontFamily{U}{matha}{\hyphenchar\font45}
\DeclareFontShape{U}{matha}{m}{n}{
      <5> <6> <7> <8> <9> <10> gen * matha
      <10.95> matha10 <12> <14.4> <17.28> <20.74> <24.88> matha12
      }{}

\newcommand{\bland}{\mathbin{
  \raisebox{.1ex}{%
    \rotatebox[origin=c]{-90}{\usefont{U}{matha}{m}{n}\symbol{\string"CE}}}}}
\newcommand{\blor}{\mathbin{
  \raisebox{.1ex}{%
    \rotatebox[origin=c]{90}{\usefont{U}{matha}{m}{n}\symbol{\string"CE}}}}}

% The colorcode environment lets us use colour markup inside verbatim blocks.
\DefineVerbatimEnvironment{colorcode}%
        {Verbatim}{fontsize=\small,commandchars=\\\{\}}

\DefineVerbatimEnvironment{bcolorcode}%
        {BVerbatim}{fontsize=\small,commandchars=\\\{\}}

% Visually Separate figures from text using a line.
\def\topfigrule{\vskip1ex\noindent\dotfill}
\def\botfigrule{\noindent\dotfill\vskip1ex}

%%% Local Variables: 
%%% mode: latex
%%% TeX-master: "thesis.tex"
%%% End: 
