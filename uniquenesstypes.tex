\chapter{Uniqueness types}
\label{chap:uniqueness-types}

While \LO{} is through and through a pure functional language, it may
occasionally prove useful to express certain algorithms in an
imperative style.  In such cases, we may also want to update arrays
\textit{in-place}, that is, with a static guarantee that the operation
will not require any additional memory allocation, such as copying the
array.  In order to perform such updates without violating referential
transparency, we need to know that no other references to the array
exist, or at least that they will not be used.  In LO{}, this is done
through a type system feature known as \textit{uniqueness types},
similar to, although simpler, than the one found in
Clean~\cite{clean-uniqueness-types}\cite{barendsen1996uniqueness}.
Alongside a (relatively) simple aliasing analysis in the type checker,
this is sufficient to determine a compile time whether an in-place
modification is safe.

The simplest way to introduce uniqueness types are through examples.
To that end, let us consider the following function definition.

\begin{colorcode}
fun \emp{*}[int] modify(\emp{*}[int] a, int i, int x) =
  let b = a with [i] <- a[i] + x in
  b
\end{colorcode}

The function \texttt{modify(a,i,x)} returns \texttt{a}, but where the
element at index \texttt{i} has been increased by \texttt{x}.  Note
the \emp{asterisks}: In the parameter declaration \texttt{*[int] a},
this means that the function \texttt{modify} has been given
``ownership'' of the array \texttt{a}, meaning that it can do whatever
it wants with its contents - the caller will never reference it again.
In particular, it can modify the element at index \texttt{i} without
first copying the array.  Furthermore, the return value of
\texttt{modify} is also unique - this means that there are no other
references to the underlying memory, and the caller can use it as it
sees fit.

Let us consider a call to \texttt{modify}, which might look as
follows.

\begin{colorcode}
let b = modify(a, i, x) in
...
\end{colorcode}

Under which circumstances is this call valid?  Two things must hold:
\begin{enumerate}
\item The type of \texttt{a} must be \texttt{*[int]}, of course.

\item Neither \texttt{a} or any variable that \textit{aliases}
  \texttt{a} may be used on any execution path following the call to
  \texttt{modify}.
\end{enumerate}

In general, when a value is passed as a unique-typed argument in a
function call, we consider that value to be \textit{consumed}, and
neither it nor any of its aliases can be used again.  Otherwise, we
would break the contract that gives the function liberty to manipulate
the argument however it wants.  Note that it is the type in the
argument declaration that must be unique - it is permissible to pass a
unique-typed variable as a non-unique argument (that is, a unique type
is a subtype of the corresponding nonunique type).

A variable aliases \texttt{a} if it may share the same memory.  As the
most trivial case, after evaluating the binding \texttt{let b = a},
the variable \texttt{b} will alias \texttt{a}.  As another example, if
we extract a row from a two-dimensional array, the row will alias its
source.  \Cref{sec:l0-sharing} will cover sharing and sharing analysis
in greater detail.

Let us consider the definition of a function returning a unique array:

\begin{colorcode}
fun *[int] f([int] a) = \(e\)
\end{colorcode}

Note that the argument, \texttt{a}, is non-unique, and hence we cannot
modify it.  But there is another restriction: \texttt{a} must not be
aliased to our return value, as the uniqueness contract requires us to
ensure that there are no other references to the unique return value.
This requirement would be violated if we permitted the return value in
a unique-returning function to alias its (non-unique) parameters.

We can crystallise these observations as three principal rules:

\begin{description}
\item[Uniqueness Rule 1] When a value is passed in the place of a
  unique parameter in a function call, or used as the source in a
  \texttt{let-with} expression, neither that value, nor any value that
  aliases it, may be used on any execution path following the function
  call.

\item[Uniqueness Rule 2] If a function definition is declared to
  return a unique value, the return value (that is, the result of the
  body of the function) must not share memory with any non-unique
  arguments to the function.

\item[Uniqueness Rule 3] If a function call yields a unique
  return value, the caller has exclusive access to that value.  The
  value may not share memory with any variable used in any execution
  path following the function call.
\end{description}

Finally, it is worth noting that everything in this chapter is used as
part of a static analysis.  \textit{All} violations of uniqueness
constraints will be discovered at compile-time (in fact, during
type-checking), thus leaving the code generator and runtime system at
liberty to exploit them for low-level optimisation.

\subsection{Sharing analysis}
\label{sec:l0-sharing}

Whenever the memory regions for two values overlap, we say that they
are \textit{aliased}, or that \textit{sharing} is present.  As an
example, if you have a two-dimensional array \texttt{a} and extract
its first row as the one-dimensional array \texttt{b}, we say that
\texttt{a} and \texttt{b} are aliased.  While the \LO{} compiler may
do a deep copy if it wishes, it is not required, and this operation
thus holds the potential for sharing memory.  Sharing analysis is
necessarily conservative, and merely imposes an upper bound on the
amount of sharing happening at runtime.  The sharing analysis in \LO{}
has been carefully designed to make the bound as tight as possible,
but still easily computable.

In \LO{}, the only values that can have any sharing are arrays -
everything else is considered ``primitive''.  Tuples are special, in
that they are not considered to have any identity beyond their
elements.  Therefore, when we store sharing information for a
tuple-typed expression, we do it for each of its element types, rather
than the tuple value as a whole.

To be precise, sharing information for an expression $e$, written
$\aliases{e}$, can take one of two forms:

\begin{enumerate}
\item $l$, where $l$ is a subset of the variables in scope at $e$.
  This means that $e$ may share data with some of the variables in
  $l$.

\item $\langle s_{1}, \ldots, s_{n} \rangle$, which requires that the
  type of $e$ is a tuple $\{t_{1}, \ldots, t_{n}\}$, and denotes that
  the sharing of the $i$th component is $s_{i}$.
\end{enumerate}

We need a way to combine sharing information.  The typical case is for
computing sharing information for the expression \texttt{if c then e1
  else e2}, where the sharing of the resulting value is the
``combination'' of the sharing in both \texttt{e1} and \texttt{e2}.
We make this combination precise by the associative, commutative
operation $s_{1} \oplus s_{2}$, which is defined by the following
equation.

\begin{align*}
  l_{1} \oplus l_{2} &= l_{1} \cup l_{2} \\
  l \oplus \langle s_{1}, \ldots, s_{n} \rangle &= \langle l \oplus s_1, \ldots, l \oplus s_n \rangle \\
   \langle s_{1}, \ldots, s_{n} \rangle \oplus l &= \langle l \oplus s_1, \ldots, l \oplus s_n \rangle \\
  \langle s_{1}, \ldots, s_{n} \rangle \oplus \langle s_{n+1}, \ldots, s_{2n} \rangle &= \langle s_{1} \oplus s_{n+1}, \ldots, s_{n} \oplus s_{2n} \rangle \\
\end{align*}

Now we can define
\[
\aliases{\texttt{if c then e1 else e2}} = \aliases{\texttt{e1}} \oplus \aliases{\texttt{e2}}.
\]
Note that $(\oplus, \emptyset)$ constitutes a monoid - this is
exploited in the \LO{} compiler.  We will often treat sharing
information as a set and write things such as $\forall
v\in\aliases{e}.p(v)$ -- in these cases, the set elements are all
variables contained anywhere in the sharing information.

Aliasing is transitive -- if $v\in\aliases{e}$, then
$v'\in\aliases{e}\Rightarrow v\in\aliases{v'}$.  Aliasing is mostly
intuitive, but here's a few rules describing how sharing is propagated
by various expressions.

\begin{align*}
  \aliases{\texttt{\textit{e}}} &= \emptyset & (\text{Whenever $e$ has a basic type}) \\
  \aliases{\texttt{copy(\textit{e})}} &= \emptyset \\
  \aliases{\texttt{if \textit{c} then ${e}_{1}$ else ${e}_{2}$}} &= \aliases{e_{1}} \oplus \aliases{e_{2}} \\
\end{align*}

The rule for function application is more complicated.  To begin with,
and this was indeed the original rule in \LO{}, we can state that the
return value of a function call aliases all of its arguments.

\begin{align*}
  \aliases{\textit{f}(e_{1}, \ldots, e_{n})} &= \bigcup_{1 \leq i \leq n} \aliases{e_{i}} & \textit{(--Too restrictive!)}
\end{align*}

However, it turns out that this is far too restrictive.  Consider a
call \texttt{f1(a)} to the function \texttt{f1} whose type is shown on
\cref{fig:unique-arguments} - if the return value aliased the argument
\texttt{a}, then we could never use the return value at all, as it
would alias something that has been consumed, namely the parameter
\texttt{a}.  Hence, a first elaboration is that the return value
should only alias those function arguments that are not consumed:

\begin{align*}
  \aliases{\textit{f}(e_{1}, \ldots, e_{n})} &= \bigcup_{1 \leq i \leq n, \text{$e_{i}$ is not consumed}} \aliases{e_{i}}  & \textit{(--Still too restrictive!)}
\end{align*}

The argument for the soundness of this rule is as follows: even if the
return value may at runtime alias a consumed argument, we do not need
to record it, as that argument will never be accessed elsewhere.

Unfortunately, the above rule is still too restrictive, as can be
illustrated by function \texttt{f2} from \cref{fig:unique-arguments}.
Consider a call \texttt{f2(a)} - by the above rule, the return value
would be aliased to \texttt{a}, which would violate Uniqueness Rule 3,
as \texttt{a} may be used again.

Hence, we add another elaboration, wherein the alias set is empty if
the return value is unique.

\begin{align*}
  \aliases{f(e_{1}, \ldots, e_{n})} &=
  \begin{cases}
    \emptyset & \mbox{If $f$ returns a unique value}\\
    \bigcup_{\overset{1 \leq i \leq n}{\text{$e_{i}$ is not consumed}}} \aliases{e_{i}} & \mbox{Otherwise}
  \end{cases}
\end{align*}

The final rule is essentially correct, except that it ignores tuples.
As mentioned earlier, sharing information for tuples is represented
element-wise.  Hence, we can simply apply the above rule piecewise for
each element in the tuple.

\begin{figure}
\begin{center}
\begin{bcolorcode}
fun [int] f1(*[int] a) = ...

fun *[int] f2([int] a) = ...
\end{bcolorcode}
\end{center}
\caption{Unique arguments}
\label{fig:unique-arguments}
\end{figure}

\subsection{Tracking uniqueness}
\label{subsec:l0-tracking-uniqueness}

If the type of an array parameter is preceded by a single asterisk, it
denotes that the array is unique - that it will never be reused
outside of the current function.  The source operand to a let-with
\textit{must} be unique.  If it is not, it is reported as a type
error.  Let-with and function calls are the only places in which
variable consumption can happen.  As a first example, let us consider
a function that replaces the value at a given position in an integer
array.

\begin{colorcode}
  fun *[int] replace(*[int] arr, int i, int x) =
    let arr[i] = x in arr
\end{colorcode}

The type of this function expresses the fact that it consumes its
array argument, and also returns a unique array.  This permits
composition - \texttt{replace(replace(a, i1, x), i2, y)} is a valid
application.  Defining \texttt{replace} as
\begin{colorcode}
  fun [int] replace2(*[int] arr, int i, int x) =
    let arr[i] = x in arr
\end{colorcode}
would still be type correct (a unique array can be used anywhere a
nonunique is expected), but the composition
\texttt{replace2(replace2(a, i1, x), i2, y)} would no longer be well
typed.

Checking that uniqueness invariants are being upheld is far subtler
than normal type checking.  In particular, detailed sharing analysis
has to be performed, in order to ensure that after an array $a$ is
modified, it becomes an error to use any value that may refer to
(parts of) the old value of the array.  Whenever we consume a variable
$a$, we mark as inaccessible all of its aliases.

\begin{colorcode}
  let b = a in               // Now \(\texttt{b}\in\aliases{\texttt{a}}\).
  let c = a with [i] <- x in // \(\forall{}v\in\aliases{\texttt{a}}\Rightarrow\textrm{Mark \(v\) as consumed.}\)
  b                          // Error, because \(\texttt{b}\in\aliases{\texttt{a}}\)!
\end{colorcode}

A key principle is that of \textit{sequence points} that lexically
checkpoint the use of variables.  As an exampe, assume that we are
given a function \texttt{f} of type \texttt{*[int] -> int}.  That is,
\texttt{f} consumes an array and returns an integer.  The expression
\begin{colorcode}
  f(a) + a[i]
\end{colorcode}
is invalid because a consumption and observation of the same variable
happens within the same \textit{sequence}.  It is valid for a sequence
to contain multiple observations of the same variable, but if a
variable is consumed, that must be the only occurence of the variable
(or any of its aliases) within the sequence.  Binding constructs
(lets, let-withs and loops) create sequence points that delimit
sequences.  If we rewrite the expression to cordon the consumption
into its own sequence, all will be well.
\begin{colorcode}
  let c = a[i] // Since a[i] is of primitive type,
               // c does not alias a.
  in f(a) + c
\end{colorcode}

The reason for this rule is to enable simpler code generation, as any
necessary order of operations is evident in the code.  It does require
a certain amount of care when doing program transformations, as for
example copy propagation may result in invalid programs.

In the previous examples, function arguments that were consumed were
all simple variables, making it easy to describe what was being
consumed.  But in general, we might have an expression
\begin{colorcode}
  replace(e, i, x)
\end{colorcode}
where \texttt{e} is some arbitrary expression.  In this case, we mark
as consumed all variables in $\aliases{\texttt{e}}$.

Constant, literal arrays are not considered unique, as the compiler
may put them in read-only memory and return the same reference every
time they are accessed.  For example, the following program is
invalid.
\begin{colorcode}
  fun [int] fibs(int i, int x) =
    let a = [1, 1, 2, 3, 5, 8, 13] in
    let a[i] = x in a
\end{colorcode}
Since \texttt{a} is not unique, its use in the let-with is a type
error.  However, we can use \texttt{copy} to create a unique duplicate
of the array.
\begin{colorcode}
  fun [int] fibs(int i, int x) =
    let a = copy([1, 1, 2, 3, 5, 8, 13]) in
    let a[i] = x in a
\end{colorcode}

If we have a function such as
\begin{colorcode}
  fun int f(*[int] a, int x) = x
\end{colorcode}
then it is not valid to curry it in such a way that we provide values
for the consumed parameters.  For example, \texttt{map(f (a), b)}
would be an error.  The reason for this is that \texttt{t} may be
called an arbitrary number of times during the mapping, but \texttt{a}
can only be consumed once.

%%% Local Variables: 
%%% mode: latex
%%% TeX-master: "thesis.tex"
%%% End: 
