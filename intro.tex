\chapter{Introduction}

For practically the entire lifetime of the electronic computer,
programmers have been used to an exponential growth in commonly
available computing power.  Until around 2006, this directly
manifested itself as improvements to sequential performance, although
physical limits made it un-economical (or impossible) for this trend
to continue.  These days, hardware designers are making their machines
increasingly \textit{parallel}: rather than speeding up the individual
processors, as happened previously, \textit{more} processors, or more
\textit{specialised} processors, are added.  Thus, while computing
power is still growing, it has become increasingly necessary to write
programs that are parallel in order to take full advantage of modern
hardware.

One interesting development is the commoditization of massively
parallel vector processors in the form of graphics cards.  While
hardware acceleration of graphics became commonplace in the 90s, it
was not until the rise of CUDA and OpenCL in 2006 that
\textit{General-Purpose computing on Graphics Processing Units}
(GPGPU) began to move into the mainstream.

... \fxnote{Insert more}

Throughout this thesis, I will often refer to a vaguely defined
``programmer'', as well as ascribe various motives and expectations to
this nebulous being.  While \LO{} is intended as an intermediate
language, and in the end is intended as a target language by compilers
for higher-level languages, it has a well-defined human-readable (and
writable) syntax, and can be programmed directly.  Indeed, all extant
\LO{} programs have been written by hand.  Thus, when I refer to ``the
programmer'', I can mean either an actual human, or a compiler
generating \LO{} code.  For my purpose, these will have identical
motives, although a human programmer may complain somewhat more
vocally the lack of syntactical niceties in the language.

\section{Contributions}

\fxnote{Surely I have contributed something?}

Parts of this thesis, in particular the core of the fusion algorithm
in \cref{chap:fusion}, has been previously published as

\begin{quote}
\fullcite{Henriksen:2013:TGA:2502323.2502328}
\end{quote}

\section{Report outline}

The remainder of the report is structured as follows.
\Cref{chap:l0language,chap:uniqueness-types} will introduce the
programmer-visible part of \LO{} and serves as a language reference.
\Cref{chap:internal} presents a slight modification of the external
language, that makes it more amenable to transformation and
optimisation.  \Cref{chap:hoisting} discusses loop-hoisting in the
context of \LO{}.  \Cref{chap:fusion} covers \textit{loop fusion}, an
important optimisation, while
\cref{chap:fusion-enabling-soac-transformations,chap:hindrance-removal}
cover transformations that enable other optimisations (although
particularly fusion).

\section{Notation}

%%% Local Variables:
%%% mode: latex
%%% TeX-master: "thesis"
%%% End:
