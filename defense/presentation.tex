\documentclass{beamer}
\usepackage[utf8x]{inputenc}
\usepackage[T1]{fontenc}
\usepackage[english]{babel}
\usepackage{url}
\usepackage{listings}
\usepackage{adjustbox}
\usepackage{chronology}
\usepackage{algpseudocode}
\usepackage{algorithm}


% \usepackage{sfmath}
%\usepackage{lxfonts}
%\usepackage{mathdesign}
\usepackage{unicode-math}
%\setmathfont{lmmath-regular.otf}
%\setmathfont{lmmath-regular.otf}
%\setmathfont{Asana-Math.otf}
%\setmathfont{xits-math.otf}


% Graphics
\usepackage{graphicx}
%\usepackage{tikz}
%\usetikzlibrary{matrix, calc, arrows, snakes}

% Figures
\usepackage{caption}
\captionsetup{labelformat=empty,labelsep=none}

% Font
\usepackage{microtype}
%\usepackage[scaled]{beramono}
\usepackage{fontspec,xunicode}
\newfontfamily\sbmyriad{Myriad Pro Semibold}
\setsansfont{Myriad Pro}
\setmonofont{Myriad Pro}
\setbeamerfont{title}{family=\sbmyriad}
\setbeamerfont{frametitle}{family=\bf}


% Beamer theme settings
\usecolortheme{seagull}
\setbeamertemplate{itemize item}{\raisebox{0.8mm}{\rule{1.2mm}{1.2mm}}}
\usenavigationsymbolstemplate{} % no navigation buttons


\title{Master's thesis defence}
\subtitle{Exploiting functional invariants to optimise parallelism:\\ A dataflow approach}
\author{Troels Henriksen}
\date{21. February 2014}
\institute{Computer Science\\
University of Copenhagen}




% ADD "parfor" to algorithmic environment
  % declaration of the new block
  \algblock{ParFor}{EndParFor}
  % customising the new block
  \algnewcommand\algorithmicparfor{\textbf{parfor}}
  \algnewcommand\algorithmicpardo{\textbf{do}}
  \algnewcommand\algorithmicendparfor{\textbf{end\ parfor}}
  \algrenewtext{ParFor}[1]{\algorithmicparfor\ #1\ \algorithmicpardo}
  \algrenewtext{EndParFor}{\algorithmicendparfor}


% Bitwise and and or operators
% http://tex.stackexchange.com/questions/39313/double-nested-logical-and-and-logical-or-symbols
\DeclareFontFamily{U}{matha}{\hyphenchar\font45}
\DeclareFontShape{U}{matha}{m}{n}{
      <5> <6> <7> <8> <9> <10> gen * matha
      <10.95> matha10 <12> <14.4> <17.28> <20.74> <24.88> matha12
      }{}

\newcommand{\bland}{\mathbin{
  \raisebox{.1ex}{%
    \rotatebox[origin=c]{-90}{\usefont{U}{matha}{m}{n}\symbol{\string"CE}}}}}
\newcommand{\blor}{\mathbin{
  \raisebox{.1ex}{%
    \rotatebox[origin=c]{90}{\usefont{U}{matha}{m}{n}\symbol{\string"CE}}}}}





\begin{document}
% Welcome
% Both a defence and this is what I do
\frame{\titlepage}

% % Nice diagram showing how the talk will progress
% \begin{frame}
%   \frametitle{Agenda}
%   \tableofcontents
% \end{frame}

\begin{frame}
  \frametitle{Free lunches}

\begin{itemize}
\item Moores law still in effect, and will be for a while..
\item ... but we no longer get many increases in sequential
  performances.
\item Modern performance increases are in the form of parallelism.
\item The most parallel machines are massively parallel vector
  processors, with commodity GPUs being particularly interesting due
  to their ubiquity.
\end{itemize}

\end{frame}

\begin{frame}
  \frametitle{GPGPU}

\begin{itemize}
\item GPUs were popularised in the 90s for graphics processing.
\item Graphics is inherently parallel, so for cost reasons, the GPU
  hardware was very parallel as well.
\item In roughly 2006, GPGPU began to take off with CUDA and then
  OpenCL.
\item GPGPU is now widespread, but still very difficult.
\end{itemize}
\end{frame}

\begin{frame}
  \frametitle{GPGPU, continued}

\begin{itemize}
\item OpenCL and CUDA both very low-level.
\item No modularisation if you want performance.
\item Libraries can contain optimised primitives... modular
  performance still tricky.
\item So we need a language.
\end{itemize}
\end{frame}

\end{document}

%%% Local Variables:
%%% mode: latex
%%% TeX-engine: luatex
%%% End: